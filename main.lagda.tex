\documentclass{article}[9pt]

\usepackage{amssymb, amsbsy, amsmath, amscd,
            latexsym, theorem, eepic, enumerate}
\usepackage{agda}
\usepackage{geometry}
\usepackage{eulervm, charter}
\usepackage{stmaryrd}

\iffalse
\geometry{papersize={6in,9in}}
\oddsidemargin 0.0in
\evensidemargin -0.5in
\textwidth 4.5in
\headheight 0.125in
\headsep 0.125in
\topmargin -0.25in
\textheight 7.0in
\fi

\iffalse
\usepackage{fontspec}

\newfontfamily{\AgdaSerifFont}{Charter}
\newfontfamily{\AgdaSansSerifFont}{Charter}
\newfontfamily{\AgdaTypewriterFont}{Charter}
\renewcommand{\AgdaFontStyle}[1]{{\AgdaSansSerifFont{}#1}}
\renewcommand{\AgdaKeywordFontStyle}[1]{{\AgdaSansSerifFont{}#1}}
\renewcommand{\AgdaStringFontStyle}[1]{{\AgdaTypewriterFont{}#1}}
\renewcommand{\AgdaCommentFontStyle}[1]{{\AgdaTypewriterFont{}#1}}
\renewcommand{\AgdaBoundFontStyle}[1]{\textit{\AgdaSerifFont{}#1}}
\fi

\usepackage{unicode-math}
\setmathfont{XITS Math}


\usepackage{newunicodechar}
\newunicodechar{𝐶}{\ensuremath{\mathit{C}}}
\newunicodechar{𝑡}{\ensuremath{\mathit{t}}}
\newunicodechar{𝑥}{\ensuremath{\mathit{x}}}
\newunicodechar{𝑇}{\ensuremath{\mathit{T}}}
\newunicodechar{𝑚}{\ensuremath{\mathit{m}}}
\newunicodechar{𝑠}{\ensuremath{\mathit{s}}}
\newunicodechar{𝑉}{\ensuremath{\mathit{V}}}
\newunicodechar{𝑎}{\ensuremath{\mathit{a}}}
\newunicodechar{𝑟}{\ensuremath{\mathit{r}}}
\newunicodechar{𝑅}{\ensuremath{\mathit{R}}}
\newunicodechar{𝑒}{\ensuremath{\mathit{e}}}
\newunicodechar{𝑛}{\ensuremath{\mathit{n}}}
\newunicodechar{𝑧}{\ensuremath{\mathit{z}}}
\newunicodechar{𝑣}{\ensuremath{\mathit{v}}}
\newunicodechar{𝑝}{\ensuremath{\mathit{p}}}
\newunicodechar{𝐴}{\ensuremath{\mathit{A}}}
\newunicodechar{ℓ}{\ensuremath{\mathnormal{\ell}}}
\newunicodechar{→}{\ensuremath{\mathnormal{\to}}}
\newunicodechar{₁}{\ensuremath{\mathit{_1}}}
\newunicodechar{₂}{\ensuremath{\mathit{_2}}}
\newunicodechar{₃}{\ensuremath{\mathit{_3}}}
\newunicodechar{₄}{\ensuremath{\mathit{_4}}}
\newunicodechar{∅}{\ensuremath{\mathnormal{\varnothing}}}
\newunicodechar{⊹}{\ensuremath{\mathnormal{\hermitmatrix}}}
\newunicodechar{⊕}{\ensuremath{\mathnormal{\oplus}}}
\newunicodechar{∘}{\ensuremath{\mathnormal{\circ}}}
\newunicodechar{⟦}{\ensuremath{\mathnormal{\llbracket}}}
\newunicodechar{⟧}{\ensuremath{\mathnormal{\rrbracket}}}
\newunicodechar{𝒾}{\ensuremath{\mathcal{i}}}
\newunicodechar{𝒹}{\ensuremath{\mathcal{d}}}
\newunicodechar{⊚}{\ensuremath{\mathnormal{\circledcirc}}}
\newunicodechar{≡}{\ensuremath{\mathnormal{\equiv}}}
\newunicodechar{⊔}{\ensuremath{\mathnormal{\sqcup}}}
\newunicodechar{σ}{\ensuremath{\mathnormal{\sigma}}}
\newunicodechar{τ}{\ensuremath{\mathnormal{\tau}}}
\newunicodechar{μ}{\ensuremath{\mathnormal{\mu}}}
\newunicodechar{π}{\ensuremath{\mathnormal{\pi}}}
\newunicodechar{β}{\ensuremath{\mathnormal{\beta}}}
\newunicodechar{η}{\ensuremath{\mathnormal{\eta}}}
\newunicodechar{λ}{\ensuremath{\mathnormal{\lambda}}}
\newunicodechar{⇒}{\ensuremath{\mathnormal{\Rightarrow}}}
\newunicodechar{⇛}{\ensuremath{\mathnormal{\Rrightarrow}}}
\newunicodechar{𝓈}{\ensuremath{\mathcal{s}}}
\newunicodechar{𝓉}{\ensuremath{\mathcal{t}}}
\newunicodechar{𝒻}{\ensuremath{\mathcal{f}}}
\newunicodechar{𝒞}{\ensuremath{\mathcal{C}}}
\newunicodechar{𝒟}{\ensuremath{\mathcal{D}}}
\newunicodechar{ℰ}{\ensuremath{\mathcal{E}}}
\mathchardef\mhyphen="2D


%\newunicodechar{}{\ensuremath{\mathit{}}}


\newcommand{\blank}{\mathord{\hspace{1pt}\text{--}\hspace{1pt}}}

\newcommand{\bkt}[1]{\left(#1\right)}
\newcommand{\lbkt}[1]{\left\llbracket#1\right\rrbracket}

\begin{document}

\section{Objective Metatheory}

\iffalse
Some code:
\begin{code}
{-# OPTIONS --without-K #-}

open import Agda.Builtin.String

-- A comment with some TeX ligatures:
-- --, ---, ?`, !`, `, ``, ', '', <<, >>.

Θ₁ : Set → Set
Θ₁ = λ A → A

a-name-with--hyphens : ∀ {A : Set} → A → A
a-name-with--hyphens ff--fl = ff--fl

ffi : String
ffi = "--"
\end{code}
Note that the code is indented.
\fi

As a novice, I learned type theory by way of teaching a summer course which
involved the students implementing their own proof assistants. As I was
preparing the lecture on substitution, I noticed an error in my source material
concerning the conditions for choosing a suitable $\alpha$-equivalent form. I
then happily presented the amended definition, and was satisfied with my
correction until I inevitably discovered yet another previously overlooked
error! While covering this third rendition, a student asked \textit{`Can we ever
know that our definition is correct?'}, and I replied that we could not, as the
notion of the \emph{correctness of a definition} is tenuous.

My students had an intuitive understanding of substitution and when I pointed
out counterexamples to the previous definitions, it was clear to them that the
prior formalism did not behave as it should have. Similarly, one level up,
simply typed lambda calculus is just the language of functions and evaluation,
so the type theory itself corresponds to something intuitive. However, despite
this, there is a multitude of ways in which one could \emph{`correctly'}
implement the system. For example, one could use either De Bruijn indices or
names; this choice affects the precise statement of the $\beta$ and $\eta$ laws
employed in the system, although the two results are analogous. A more profound
implementation decision is the choice between direct and explicit substitution.
In the latter, the syntax of terms is enriched with a substitution form
$t~[~\sigma~]$ and new reduction rules are added which explain how to propagate
substitutions into each syntactic category of terms. One then asks the question:
\emph{`What, generally, is simply typed lambda calculus?'}

Categorical logic provides a satisfactory answer to this question. This
resolution begins with the axiomatisation of a certain mathematical structure
known as a \emph{contextual cartesian closed category} (CCC). A syntactic model
of STLC is to have such a structure, and, most significantly, is to be initial
among all such structures. A model of STLC, then, is precisely a language for
expressing any construction that is globally valid in any CCC. From this
perspective, type theories are not arbitrary collections of rules that can be
modified indifferently. In other words, contrary to what I indicated in my
course, there is a formal notion of what it means for an specification of STLC
to be correct!

In his thesis, Jon Sterling coined the distinction between \emph{Objective} and
\emph{Subjective} metatheoretic statements about type theory. Subjective
metatheory concerns the properties of a specific presentation of a type theory
while objective metatheory concerns properties of any initial presentation of
the appropriately structured category, given solely in terms of the
axiomatisation of such a structure and the hypothesis of initiality.

An excellent example of this distinction is the question of what is meant by a
normalisation theorem. Traditionally, type theories were endowed with an
operational semantics, i.e. a directed series of rewrite rules. A normalisation
theorem, then, asserted that the process of iterated reduction terminated. Such
a conception of normalisation is subjective (e.g. the number of rules in
question depends on whether or not one uses explicit substitution), and one
might ask what it means to formulate normalisation in the absence of a rewrite
structure.

This is achieved by shifting from operational to denotational semantics. In
denotational semantics, a recipe is provided for interpreting syntax as objects
in some mathematical domain. For example, a lambda term could be interpreted as
a set theoretic function (or, in our case, a presheaf morphism). The
denotational analogue of evaluating by way of repeated reductions is
\emph{Normalisation by Evaluation}, which is a seemingly \emph{reduction-free}
process that is to be thought of as antithetical to the process of producing a
computational trace. NbE is suited to being phrased objectively because the
mathematical domains in question are nothing but instances of CCCs and the
interpretation of syntax is uniquely furnished by its initiality.

In order to obtain a NbE algorithm, one first defines a notion of normal forms.
This is done by mutual induction with a notion of neutral terms, which are to be
thought of as \emph{generalised variables} is the sense of being terms that are
maximally blocked from participating in evaluation by a lack of information. The
normal and neutral forms of type $A$ depend on a context $\Gamma$ and one thus
obtains a presheaves $\mathsf{Nf}~(\blank,A)$ and $\mathsf{Ne}~(\blank,A)$ on the
category of context renamings (provided that one has a notion of renaming a
normal/neutral form). One then defines an interpretation of the syntax of STLC
into presheaves by specifying that the base types $X$ are to be interpreted as
the preshaves $\mathsf{Nf}~(\blank,X)$. This endows an interpretation $\lbkt{A}$
and $\lbkt{\Gamma}$ in presheaves to every type $A$ and context $\Gamma$. A term
$\Gamma \vdash t : A$, written $t : \textsf{Tm}~(\Gamma, A)$, is interpreted as a
presheaf morphism $\lbkt{t} : \lbkt{\Gamma} \to \lbkt{A}$. Note that all of this
is achieved automatically as soon as one shows that presheaves form a CCC.

Next, one constructs presheaf morphisms $q_A : \lbkt{A} \to \mathsf{Nf}~(\blank,
A)$ and $u_A : \mathsf{Ne}~(\blank, A) \to \lbkt{A}$, known as \emph{quote} and
\emph{unquote}; these extend to contexts. Finally, the context morphism
$1_\Gamma : \mathsf{Tm}~(\Gamma, \Gamma)$ consisting of variables can be given as a
list of neutral terms $1_\Gamma : \mathsf{Ne}~(\Gamma,\Gamma)$. From this, we
have: \[q_{A,\Gamma} \bkt{\lbkt{t}_\Gamma \bkt{u_{\Gamma,\Gamma}
\bkt{1_\Gamma}}} : \mathsf{Nf}~(\Gamma, A).\] This defines a function
$\mathsf{norm} : \mathsf{Tm}~(\Gamma, A) \to \mathsf{Nf}~(\Gamma, A)$.

In the subjective approach, quote and unquote are defined by induction over the
structure of $A$ (this assumes that the types of STLC have a certain form), and
the correctness of NbE (i.e. the statement that $\iota\mathsf{Nf}
\bkt{\mathsf{norm} \bkt{t}} \equiv t$) is given by a \emph{Logical Relations
Proof}. In the objective approach, however, one replaces the category of
presheaves with a more complex structured category that embeds the definitions
of quote and unquote, as well as a correctness proof. The entirety of the the
normalisation proof, then, effectively consists of showing that this more
complex structured category is a CCC; this specifies a recipe for structurally
building up a proof of a normalisation theorem. This purely categorical argument
is known as a \emph{Categorical Glueing Proof} for normalisation.

\section{The Structure of Non-Dependent Type Theory}

Type theories tend to have a specific structure, resulting in the formal
specification of several notions of a \emph{categorical model of type theory},
such as Dybjer's \emph{Categories with Families}. The development to this
project did not begin with such a structure in mind, but as I was formalising
the category of twisted glueings, it became apparent that an organising theory
was necessary. Moreover, enough code had been written such that several
recurring data patterns naturally began to emerge, and formulating these
structures led to a substantial de-boilerplating of the codebase.

The result is a somewhat novel conception of a categorical model of type theory
which we refer to as a \textbf{Contextual Category} \emph{(in the sense of
Astra)}. This structure is equivalent to, albeit presented differently from that
of, a \emph{Cartesian Multicategory}. The latter is already known to lend itself
more easily to working with type theory than CwFs, with the drawback of being
unable to describe dependent type theories, and, from the perspective of the
codebase, contextual categories emerged as the most amenable way to capture
\emph{the situation of non-dependent type theory}.

\input{lists}

\input{contextual}

\input{CCC}

\input{NbE}














\end{document}