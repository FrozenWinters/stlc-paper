\documentclass{article}[9pt]

\usepackage{amssymb, amsbsy, amsmath, amscd,
            latexsym, theorem, eepic, enumerate}
\usepackage{agda}
\usepackage{geometry}
\usepackage{eulervm, charter}
\usepackage{stmaryrd}

\iffalse
\geometry{papersize={6in,9in}}
\oddsidemargin 0.0in
\evensidemargin -0.5in
\textwidth 4.5in
\headheight 0.125in
\headsep 0.125in
\topmargin -0.25in
\textheight 7.0in
\fi

\iffalse
\usepackage{fontspec}

\newfontfamily{\AgdaSerifFont}{Charter}
\newfontfamily{\AgdaSansSerifFont}{Charter}
\newfontfamily{\AgdaTypewriterFont}{Charter}
\renewcommand{\AgdaFontStyle}[1]{{\AgdaSansSerifFont{}#1}}
\renewcommand{\AgdaKeywordFontStyle}[1]{{\AgdaSansSerifFont{}#1}}
\renewcommand{\AgdaStringFontStyle}[1]{{\AgdaTypewriterFont{}#1}}
\renewcommand{\AgdaCommentFontStyle}[1]{{\AgdaTypewriterFont{}#1}}
\renewcommand{\AgdaBoundFontStyle}[1]{\textit{\AgdaSerifFont{}#1}}
\fi

\usepackage{unicode-math}
\setmathfont{XITS Math}


\usepackage{newunicodechar}
\newunicodechar{𝐶}{\ensuremath{\mathit{C}}}
\newunicodechar{𝑡}{\ensuremath{\mathit{t}}}
\newunicodechar{𝑥}{\ensuremath{\mathit{x}}}
\newunicodechar{𝑇}{\ensuremath{\mathit{T}}}
\newunicodechar{𝑚}{\ensuremath{\mathit{m}}}
\newunicodechar{𝑠}{\ensuremath{\mathit{s}}}
\newunicodechar{𝑉}{\ensuremath{\mathit{V}}}
\newunicodechar{𝑎}{\ensuremath{\mathit{a}}}
\newunicodechar{𝑟}{\ensuremath{\mathit{r}}}
\newunicodechar{𝑅}{\ensuremath{\mathit{R}}}
\newunicodechar{𝑒}{\ensuremath{\mathit{e}}}
\newunicodechar{𝑛}{\ensuremath{\mathit{n}}}
\newunicodechar{𝑧}{\ensuremath{\mathit{z}}}
\newunicodechar{𝑣}{\ensuremath{\mathit{v}}}
\newunicodechar{𝑝}{\ensuremath{\mathit{p}}}
\newunicodechar{𝐴}{\ensuremath{\mathit{A}}}
\newunicodechar{ℓ}{\ensuremath{\mathnormal{\ell}}}
\newunicodechar{→}{\ensuremath{\mathnormal{\to}}}
\newunicodechar{₁}{\ensuremath{\mathit{_1}}}
\newunicodechar{₂}{\ensuremath{\mathit{_2}}}
\newunicodechar{₃}{\ensuremath{\mathit{_3}}}
\newunicodechar{₄}{\ensuremath{\mathit{_4}}}
\newunicodechar{₅}{\ensuremath{\mathit{_5}}}
\newunicodechar{₆}{\ensuremath{\mathit{_6}}}
\newunicodechar{∅}{\ensuremath{\mathnormal{\varnothing}}}
\newunicodechar{⊹}{\ensuremath{\mathnormal{\hermitmatrix}}}
\newunicodechar{⊕}{\ensuremath{\mathnormal{\oplus}}}
\newunicodechar{∘}{\ensuremath{\mathnormal{\circ}}}
\newunicodechar{⟦}{\ensuremath{\mathnormal{\llbracket}}}
\newunicodechar{⟧}{\ensuremath{\mathnormal{\rrbracket}}}
\newunicodechar{⦃}{\ensuremath{\mathnormal{\lBrace}}}
\newunicodechar{⦄}{\ensuremath{\mathnormal{\rBrace}}}
\newunicodechar{𝒾}{\ensuremath{\mathcal{i}}}
\newunicodechar{𝒹}{\ensuremath{\mathcal{d}}}
\newunicodechar{⊚}{\ensuremath{\mathnormal{\circledcirc}}}
\newunicodechar{≡}{\ensuremath{\mathnormal{\equiv}}}
\newunicodechar{⊔}{\ensuremath{\mathnormal{\sqcup}}}
\newunicodechar{σ}{\ensuremath{\mathnormal{\sigma}}}
\newunicodechar{τ}{\ensuremath{\mathnormal{\tau}}}
\newunicodechar{μ}{\ensuremath{\mathnormal{\mu}}}
\newunicodechar{π}{\ensuremath{\mathnormal{\pi}}}
\newunicodechar{β}{\ensuremath{\mathnormal{\beta}}}
\newunicodechar{η}{\ensuremath{\mathnormal{\eta}}}
\newunicodechar{ι}{\ensuremath{\mathnormal{\iota}}}
\newunicodechar{λ}{\ensuremath{\mathnormal{\lambda}}}
\newunicodechar{⇒}{\ensuremath{\mathnormal{\Rightarrow}}}
\newunicodechar{⇛}{\ensuremath{\mathnormal{\Rrightarrow}}}
\newunicodechar{⇓}{\ensuremath{\mathnormal{\Downarrow}}}
\newunicodechar{𝓈}{\ensuremath{\mathcal{s}}}
\newunicodechar{𝓉}{\ensuremath{\mathcal{t}}}
\newunicodechar{𝒻}{\ensuremath{\mathcal{f}}}
\newunicodechar{𝒞}{\ensuremath{\mathcal{C}}}
\newunicodechar{𝒟}{\ensuremath{\mathcal{D}}}
\newunicodechar{ℰ}{\ensuremath{\mathcal{E}}}
\newunicodechar{∀}{\ensuremath{\forall}}

\mathchardef\mhyphen="2D


%\newunicodechar{}{\ensuremath{\mathit{}}}


\newcommand{\blank}{\mathord{\hspace{1pt}\text{--}\hspace{1pt}}}

\newcommand{\bkt}[1]{\left(#1\right)}
\newcommand{\lbkt}[1]{\left\llbracket#1\right\rrbracket}

\title{The Objective Metatheory of Simply Typed Lambda Calculus}
\author{Astra Kolomatskaia\thanks{We acknowledge the support of the Natural
Sciences and Engineering Research Council of Canada (NSERC).
Cette recherche a \'et\'e financ\'ee par le Conseil de recherches en sciences
naturelles et en g\'enie du Canada (CRSNG).
[funding reference number CGSD3-545891-2020]}}
\date{February 2022}

\begin{document}

\maketitle

\begin{abstract}
\noindent
We develop, in detail, and formalise, in Cubical Agda, the categorical semantics
of simply typed lambda calculus and the twisted glueing argument for
normalisation developed by Altenkirch et al [1995]. This exposition features the
development of a novel notion of a categorical model of non-dependent type
theory, which we refer to as a \emph{Simple Contextual Category}. Following the
rather baroque formalisation, a rewrite is undertaken in the interest of
producing a minimal program for verifying the correctness of \emph{Normalisation
by Evaluation} by means of a computational trace, hence pioneering the notion of
\emph{Stepping by Evaluation}.
\end{abstract}

\section{Objective Metatheory}

\iffalse
Some code:
\begin{code}
{-# OPTIONS --without-K #-}

open import Agda.Builtin.String

-- A comment with some TeX ligatures:
-- --, ---, ?`, !`, `, ``, ', '', <<, >>.

Θ₁ : Set → Set
Θ₁ = λ A → A

a-name-with--hyphens : ∀ {A : Set} → A → A
a-name-with--hyphens ff--fl = ff--fl

ffi : String
ffi = "--"
\end{code}
Note that the code is indented.
\fi

As a novice, I learned type theory by way of teaching a summer course which
involved the students implementing their own proof assistants. As I was
preparing the lecture on substitution, I noticed an error in my source material
concerning the conditions for choosing a suitable $\alpha$-equivalent form. I
then happily presented the amended definition, and was satisfied with my
correction until I inevitably discovered yet another previously overlooked
error! While covering this third rendition, a student asked \textit{`Can we ever
know that our definition is correct?'}, and I replied that we could not, as the
notion of the \emph{correctness of a definition} is tenuous.

My students had an intuitive understanding of substitution and when I pointed
out counterexamples to the previous definitions, it was clear to them that the
prior formalism did not behave as it should have. Similarly, one level up,
simply typed lambda calculus is just the language of functions and evaluation,
so the type theory itself corresponds to something intuitive. However, despite
this, there is a multitude of ways in which one could \emph{`correctly'}
implement the system. For example, one could use either De Bruijn indices or
names; this choice affects the precise statement of the $\beta$ and $\eta$ laws
employed in the system, although the two results are analogous. A more profound
implementation decision is the choice between direct and explicit substitution.
In the latter, the syntax of terms is enriched with a substitution form
$t~[~\sigma~]$ and new reduction rules are added which explain how to propagate
substitutions into each syntactic category of terms. One then asks the question:
\emph{`What, generally, is simply typed lambda calculus?'}

Categorical logic provides a satisfactory answer to this question. This
resolution begins with the axiomatisation of a certain mathematical structure
known as a \emph{contextual cartesian closed category} (CCC category). A
syntactic model of STLC is to have such a structure, and, most significantly, is
to be initial among all such structures. A model of STLC, then, is precisely a
language for expressing any construction that is globally valid in any CCC
category. From this perspective, type theories are not arbitrary collections of
rules that can be modified indifferently. In other words, contrary to what I
indicated in my course, there is a formal notion of what it means for an
specification of STLC to be correct!

In his thesis, Jon Sterling coined the distinction between \emph{Objective} and
\emph{Subjective} metatheoretic statements about type theory. Subjective
metatheory concerns the properties of a specific presentation of a type theory
while objective metatheory concerns properties of any initial presentation of
the appropriately structured category, given solely in terms of the
axiomatisation of such a structure and the hypothesis of initiality.

An excellent example of this distinction is the question of what is meant by a
normalisation theorem. Traditionally, type theories were endowed with an
operational semantics, i.e. a directed series of rewrite rules. A normalisation
theorem, then, asserted that the process of iterated reduction terminated. Such
a conception of normalisation is subjective (e.g. the number of rules in
question depends on whether or not one uses explicit substitution), and one
might ask what it means to formulate normalisation in the absence of a rewrite
structure.

Sterling points out that the subjective conception of normal forms typically
comes in the form of a \emph{property} on terms, i.e. that there is a judgment
$\vdash \textsf{M}~value$. This notion however, is not invariant under
definitional equality, as we may not conclude, for example, that $\vdash
\bkt{\bkt{\lambda x. x}~5}~value$, making an objective treatment untenable. One
instead shifts to thinking of normal forms as being a \emph{structure} that can
be defined on \emph{any} theory of the type system (in our case, on any CCC
category). That is, we define on any CCC category a family $\mathsf{Nf}$ of
normal forms and an inclusion $\iota\mathsf{Nf}$ of normal forms into terms. A
normalisation theorem, then, states that for any term $t$, one can construct a
normal form $\mathsf{norm}~t$, such that $\iota\mathsf{Nf}~\bkt{\mathsf{norm}
~t} \equiv t$.

Upon examining the traditional utility of a normalisation theorem, one sees that
additional requirements are to be imposed. Subjectively, when paired with a
confluence theorem, normalisation establishes the decidability of term equality.
Further, by induction on normal forms, one can establish canonicity and
soundness. Normalisation is thus a metatheoretic goldmine for type systems with
this property, and we should require the objective notion of a normalisation
theorem to be similarly fruitful. What this means is that the notion
$\mathsf{Nf}$ considered above is \emph{to be useful}. For example,
$\mathsf{Nf}$ should have decidable equality, as that would imply the same for
any theory in which we can prove a normalisation result. With these
restrictions, one will, in general, only hope to prove such a result only for
initial theories.

In practice, this objective approach corresponds to the shift from
\emph{operational} to \emph{denotational} semantics. Denotationally,
normalisation is achieved by \emph{reduction-free} approaches that generally
fall under the heading of \emph{Normalisation by Evaluation} (NbE). In the
literature, much was written about NbE from a non-categorical perspective, until
the seminal 1995 paper by Altenkirch, Hofmann, and Striecher, \emph{Categorical
reconstruction of a reduction free normalisation proof}, which derived NbE and a
proof of its correctness directly from categorical principles. Although this is
not necessarily evident in their paper, the approach taken by the authors is
objective and can be gently translated into a proof of normalisation for any
initial CCC category.

The primary contribution of this paper is a Cubical Agda formalisation of the
aforementioned work. In the course of this formalisation, many concepts were
elaborated and clarified, leading to the organising notion of a \emph{simple
contextual category}. Our goal in what follows is to give a detailed exposition
of the ideas in Altenkirch's paper and the general notion of NbE. Finally, as
a bonus to having all of these ideas written down in code, we will refactor
our project as to implement an \emph{NbE powered stepper}.

\iffalse
This is achieved by shifting from operational to denotational semantics. In
denotational semantics, a recipe is provided for interpreting syntax as objects
in some mathematical domain. For example, a lambda term could be interpreted as
a set theoretic function (or, in our case, a presheaf morphism). The
denotational analogue of evaluating by way of repeated reductions is
\emph{Normalisation by Evaluation}, which is a seemingly \emph{reduction-free}
process that is to be thought of as antithetical to the process of producing a
computational trace. NbE is suited to being phrased objectively because the
mathematical domains in question are nothing but instances of CCCs and the
interpretation of syntax is uniquely furnished by its initiality.

In order to obtain a NbE algorithm, one first defines a notion of normal forms.
This is done by mutual induction with a notion of neutral terms, which are to be
thought of as \emph{generalised variables} is the sense of being terms that are
maximally blocked from participating in evaluation by a lack of information. The
normal and neutral forms of type $A$ depend on a context $\Gamma$ and one thus
obtains a presheaves $\mathsf{Nf}~(\blank,A)$ and $\mathsf{Ne}~(\blank,A)$ on the
category of context renamings (provided that one has a notion of renaming a
normal/neutral form). One then defines an interpretation of the syntax of STLC
into presheaves by specifying that the base types $X$ are to be interpreted as
the preshaves $\mathsf{Nf}~(\blank,X)$. This endows an interpretation $\lbkt{A}$
and $\lbkt{\Gamma}$ in presheaves to every type $A$ and context $\Gamma$. A term
$\Gamma \vdash t : A$, written $t : \textsf{Tm}~(\Gamma, A)$, is interpreted as a
presheaf morphism $\lbkt{t} : \lbkt{\Gamma} \to \lbkt{A}$. Note that all of this
is achieved automatically as soon as one shows that presheaves form a CCC.

Next, one constructs presheaf morphisms $q_A : \lbkt{A} \to \mathsf{Nf}~(\blank,
A)$ and $u_A : \mathsf{Ne}~(\blank, A) \to \lbkt{A}$, known as \emph{quote} and
\emph{unquote}; these extend to contexts. Finally, the context morphism
$1_\Gamma : \mathsf{Tm}~(\Gamma, \Gamma)$ consisting of variables can be given as a
list of neutral terms $1_\Gamma : \mathsf{Ne}~(\Gamma,\Gamma)$. From this, we
have: \[q_{A,\Gamma} \bkt{\lbkt{t}_\Gamma \bkt{u_{\Gamma,\Gamma}
\bkt{1_\Gamma}}} : \mathsf{Nf}~(\Gamma, A).\] This defines a function
$\mathsf{norm} : \mathsf{Tm}~(\Gamma, A) \to \mathsf{Nf}~(\Gamma, A)$.

In the subjective approach, quote and unquote are defined by induction over the
structure of $A$ (this assumes that the types of STLC have a certain form), and
the correctness of NbE (i.e. the statement that $\iota\mathsf{Nf}
\bkt{\mathsf{norm} \bkt{t}} \equiv t$) is given by a \emph{Logical Relations
Proof}. In the objective approach, however, one replaces the category of
presheaves with a more complex structured category that embeds the definitions
of quote and unquote, as well as a correctness proof. The entirety of the the
normalisation proof, then, effectively consists of showing that this more
complex structured category is a CCC; this specifies a recipe for structurally
building up a proof of a normalisation theorem. This purely categorical argument
is known as a \emph{Categorical Glueing Proof} for normalisation.
\fi

\section{The Structure of Non-Dependent Type Theory}

Type theories tend to have a specific structure, resulting in the formal
specification of several notions of a \emph{categorical model of type theory},
such as Dybjer's \emph{Categories with Families}. The development to this
project did not begin with such a structure in mind, but as I was formalising
the category of twisted glueings, it became apparent that an organising theory
was necessary. Moreover, enough code had been written such that several
recurring data patterns naturally began to emerge, and formulating these
structures led to a substantial de-boilerplating of the codebase.

The result is a somewhat novel conception of a categorical model of type theory
which we refer to as a \emph{Simple Contextual Category}. This structure is
equivalent to, albeit presented differently from that of, a \emph{Cartesian
Multicategory}. The latter is already known to lend itself more easily to
working with type theory than CwFs, with the drawback of being unable to
describe dependent type theories, and, from the perspective of the codebase,
contextual categories emerged as the most amenable way to capture \emph{the
situation of non-dependent type theory}.

\clearpage

\input{lists}

\input{contextual}

\input{cart}

\input{CCC}

\input{NbE}














\end{document}