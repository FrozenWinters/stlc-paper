\begin{code}[hide]
{-# OPTIONS --cubical #-}

module ccc where

open import contextual

open import Cubical.Categories.Category

private
  variable
    ℓ ℓ₁ ℓ₂ ℓ₃ ℓ₄ ℓ₅ ℓ₆ : Level

-- Here is a definition of a Cartesian Closed Contextual Category
\end{code}

\section{Theories For Lambda Calculus}

It is not the case that every single type theory has a notion of function types.
One, could, for example, consider a type theory which only has base and product
types; the metatheory of such a type system is non-trivial and one could
replicate the overall yoga used in this paper to \emph{Objectively} derive a
provably correct NbE algorithm for all initial languages that roughly look like
their syntax has pairing and projections.

Similarly, it is not true that every type theory that has a notion of functions
behaves like STLC. For example, untyped lambda calculus is certainly a language
of functions. One thinks of it as a \emph{unityped} theory in which, presented
as a contextual category, we would have $\mathsf{ty} = \top$. Though it appears
that such a $\mathsf{ty}$ is incapable of describing functions, the way in which
functions very meaningfully make sense in untyped lambda calculus is that there
is a natural contextual functor between from STLC to untyped lambda calculus
given by \emph{type erasure}.

As another example, we can, in Agda, construct a contextual category whose
$\mathsf{ty}$ is the type of h-sets in some universe. The set of terms between a
context $\Gamma$ and a type $A$ is given by forming an iterated function type,
i.e. such that $\mathsf{tm}~\varnothing~A = A$, and $\mathsf{tm}~(\Gamma
\hermitmatrix B)~A = B \to (\mathsf{tm}~\Gamma~A)$ (these will also be h-sets).
In order to specify how to interpret STLC into this theory, we specify, for each
base type $X$, an h-set $\lbkt{X}$. We then have $\lbkt{A \Rightarrow B} =
\lbkt{A} \to \lbkt{B}$. Finally, since the type system that we are programming
in is sufficiently elaborate, we can extend this to an interpretation of STLC
into Cubical Agda.
\iffalse
(Of course, if we interpret a base type $X$ as the set
$\mathbb{N}$, then the successor function $\mathbb{N} \to \mathbb{N}$ will not
be the interpretation of any closed STLC term of type $X \to X$, so we are
interpreting into a much stronger, dependently typed, theory.)
\fi

Up to this point in the paper, we have been quite coy about what precisely we
mean by STLC, although we expect the reader to have vague ability to recognise
an STLC implementation when she sees one. We will proceed by giving the notion
of what it means for a contextual category to admit an interpretation of STLC,
and explain how this notion gives rise to a notion of what it means to be
\emph{an implementation of STLC}. Finally, near the end of the paper, we will
construct, according to this meaning, a suitable implementation, which one
should recognise as \emph{`being STLC'} in the \emph{ontical/pre-categorical}
sense of the term.

\begin{code}
record CCC (𝒞 : Contextual ℓ₁ ℓ₂) : Type (ℓ₁ ⊔ ℓ₂) where
  open Contextual 𝒞

  field
    _⇛_ : ty → ty → ty
    Λ : {Γ : ctx} {A B : ty} → tm (Γ ⊹ A) B → tm Γ (A ⇛ B)
    𝑎𝑝𝑝 : {Γ : ctx} {A B : ty} → tm Γ (A ⇛ B) → tm Γ A → tm Γ B

  -- Categorical app operator
  𝐴𝑝𝑝 : {Γ : ctx} {A B : ty} → tm Γ (A ⇛ B) → tm (Γ ⊹ A) B
  𝐴𝑝𝑝 t = 𝑎𝑝𝑝 (t ⟦ π ⟧) 𝑧

  field
    Λnat : {Γ Δ : ctx} {A B : ty} (t : tm (Δ ⊹ A) B) (σ : tms Γ Δ) →
      Λ t ⟦ σ ⟧ ≡  Λ ( t ⟦ W₂tms A σ ⟧)
    𝑎𝑝𝑝β : {Γ : ctx} {A B : ty} (t : tm (Γ ⊹ A) B) (s : tm Γ A) →
      𝑎𝑝𝑝 (Λ t) s ≡ t ⟦ 𝒾𝒹 Γ ⊕ s ⟧
    𝑎𝑝𝑝η : {Γ : ctx} {A B : ty} (t : tm Γ (A ⇛ B)) →
      t ≡ Λ (𝐴𝑝𝑝 t)
\end{code}

\begin{code}[hide]

  -- Some useful lemmas

  ⊕⊚ : {Γ Δ Σ : ctx} {A : ty} (σ : tms Δ Σ) (t : tm Δ A) (τ : tms Γ Δ) →
    (σ ⊕ t) ⊚ τ ≡ (σ ⊚ τ) ⊕ (t ⟦ τ ⟧)
  ⊕⊚ σ t τ =
    (σ ⊕ t) ⊚ τ
      ≡⟨ π𝑧η (σ ⊕ t ⊚ τ) ⁻¹ ⟩
    π ⊚ (σ ⊕ t ⊚ τ) ⊕ 𝑧 ⟦ σ ⊕ t ⊚ τ ⟧
      ≡⟨ (λ i → ⊚Assoc π (σ ⊕ t) τ (~ i) ⊕ ⟦⟧⟦⟧ 𝑧 (σ ⊕ t) τ (~ i)) ⟩
    π ⊚ (σ ⊕ t) ⊚ τ ⊕ 𝑧 ⟦ σ ⊕ t ⟧ ⟦ τ ⟧
      ≡⟨ (λ i → (πβ (σ ⊕ t) i ⊚ τ) ⊕ (𝑧β (σ ⊕ t) i ⟦ τ ⟧)) ⟩
    (σ ⊚ τ) ⊕ (t ⟦ τ ⟧)
      ∎

  private
    lem : {Γ Δ : ctx} {A : ty} (σ : tms Γ Δ) (t : tm Γ A) →
      ((σ ⊚ π) ⊕ 𝑧) ⊚ (𝒾𝒹 Γ ⊕ t) ≡ σ ⊕ t
    lem {Γ} σ t =
      ((σ ⊚ π) ⊕ 𝑧) ⊚ (𝒾𝒹 Γ ⊕ t)
        ≡⟨ ⊕⊚ (σ ⊚ π) 𝑧 (𝒾𝒹 Γ ⊕ t) ⟩
      σ ⊚ π ⊚ (𝒾𝒹 Γ ⊕ t) ⊕ 𝑧 ⟦ 𝒾𝒹 Γ ⊕ t ⟧
        ≡⟨ (λ i → ⊚Assoc σ π (𝒾𝒹 Γ ⊕ t) i ⊕ 𝑧β (𝒾𝒹 Γ ⊕ t) i) ⟩
      σ ⊚ (π ⊚ (𝒾𝒹 Γ ⊕ t)) ⊕ t
        ≡⟨ (λ i → σ ⊚ (πβ (𝒾𝒹 Γ ⊕ t) i) ⊕ t) ⟩
      σ ⊚ 𝒾𝒹 Γ ⊕ t
        ≡⟨ ap (_⊕ t) (𝒾𝒹R σ) ⟩
      σ ⊕ t
        ∎

  𝐴𝑝𝑝β : {Γ : ctx} {A B : ty} (t : tm (Γ ⊹ A) B) →
    𝐴𝑝𝑝 (Λ t) ≡ t
  𝐴𝑝𝑝β {Γ} {A} {B} t =
    𝑎𝑝𝑝 (Λ t ⟦ π ⟧) 𝑧
      ≡⟨ (λ i → 𝑎𝑝𝑝 (Λnat t π i) 𝑧) ⟩
    𝑎𝑝𝑝 (Λ (t ⟦ (π ⊚ π) ⊕ 𝑧 ⟧)) 𝑧
      ≡⟨ 𝑎𝑝𝑝β (t ⟦ (π ⊚ π) ⊕ 𝑧 ⟧) 𝑧 ⟩
    t ⟦ (π ⊚ π) ⊕ 𝑧 ⟧ ⟦ 𝒾𝒹 (Γ ⊹ A) ⊕ 𝑧 ⟧
      ≡⟨ ⟦⟧⟦⟧ t ((π ⊚ π) ⊕ 𝑧) (𝒾𝒹 (Γ ⊹ A) ⊕ 𝑧) ⟩
    t ⟦ ((π ⊚ π) ⊕ 𝑧) ⊚ (𝒾𝒹 (Γ ⊹ A) ⊕ 𝑧) ⟧
      ≡⟨ ap (t ⟦_⟧) (lem π 𝑧) ⟩
    t ⟦ π ⊕ 𝑧 ⟧
      ≡⟨ ap (t ⟦_⟧) 𝒾𝒹η ⟩
    t ⟦ 𝒾𝒹 (Γ ⊹ A) ⟧
      ≡⟨ 𝒾𝒹⟦⟧ t ⟩
    t
      ∎

  𝐴𝑝𝑝⟦⟧ : {Γ Δ : ctx} {A B : ty} (t : tm Δ (A ⇛ B)) (σ : tms Γ Δ) →
    𝐴𝑝𝑝 (t ⟦ σ ⟧) ≡ (𝐴𝑝𝑝 t ⟦ σ ⊚ π ⊕ 𝑧 ⟧)
  𝐴𝑝𝑝⟦⟧ t σ =
    𝐴𝑝𝑝 (t ⟦ σ ⟧)
      ≡⟨ (λ i → 𝐴𝑝𝑝 (𝑎𝑝𝑝η t i ⟦ σ ⟧)) ⟩
    𝐴𝑝𝑝 (Λ (𝐴𝑝𝑝 t) ⟦ σ ⟧)
      ≡⟨ ap 𝐴𝑝𝑝 (Λnat (𝐴𝑝𝑝 t) σ) ⟩
    𝐴𝑝𝑝 (Λ (𝐴𝑝𝑝 t ⟦ σ ⊚ π ⊕ 𝑧 ⟧))
      ≡⟨ 𝐴𝑝𝑝β (𝐴𝑝𝑝 t ⟦ σ ⊚ π ⊕ 𝑧 ⟧) ⟩
    𝐴𝑝𝑝 t ⟦ σ ⊚ π ⊕ 𝑧 ⟧
      ∎

  𝑎𝑝𝑝𝐴𝑝𝑝 : {Γ : ctx} {A B : ty} (t : tm Γ (A ⇛ B)) (s : tm Γ A) →
    𝑎𝑝𝑝 t s ≡ 𝐴𝑝𝑝 t ⟦ 𝒾𝒹 Γ ⊕ s ⟧
  𝑎𝑝𝑝𝐴𝑝𝑝 {Γ} t s =
    𝑎𝑝𝑝 t s
      ≡⟨ (λ i → 𝑎𝑝𝑝 (𝑎𝑝𝑝η t i) s) ⟩
    𝑎𝑝𝑝 (Λ (𝐴𝑝𝑝 t)) s
      ≡⟨ 𝑎𝑝𝑝β (𝐴𝑝𝑝 t) s ⟩
    𝐴𝑝𝑝 t ⟦ 𝒾𝒹 Γ ⊕ s ⟧
      ∎

  -- We finally get to the substitution law for applications;
  -- this follows from the axioms, with great difficulty.

  𝑎𝑝𝑝⟦⟧ : {Γ Δ : ctx} {A B : ty} (t : tm Δ (A ⇛ B)) (s : tm Δ A) (σ : tms Γ Δ) →
    𝑎𝑝𝑝 t s ⟦ σ ⟧ ≡ 𝑎𝑝𝑝 (t ⟦ σ ⟧) (s ⟦ σ ⟧)
  𝑎𝑝𝑝⟦⟧ {Γ} {Δ} t s σ =
    𝑎𝑝𝑝 t s ⟦ σ ⟧
      ≡⟨ ap (_⟦ σ ⟧) (𝑎𝑝𝑝𝐴𝑝𝑝 t s) ⟩
    𝐴𝑝𝑝 t ⟦ 𝒾𝒹 Δ ⊕ s ⟧ ⟦ σ ⟧
      ≡⟨ ⟦⟧⟦⟧ (𝐴𝑝𝑝 t) (𝒾𝒹 Δ  ⊕ s) σ ⟩
    𝐴𝑝𝑝 t ⟦ 𝒾𝒹 Δ ⊕ s ⊚ σ ⟧
      ≡⟨ ap (𝐴𝑝𝑝 t ⟦_⟧) (⊕⊚ (𝒾𝒹 Δ) s σ) ⟩
    𝐴𝑝𝑝 t ⟦ (𝒾𝒹 Δ) ⊚ σ ⊕ s ⟦ σ ⟧ ⟧
      ≡⟨ (λ i → 𝐴𝑝𝑝 t ⟦ 𝒾𝒹L σ i ⊕ s ⟦ σ ⟧ ⟧) ⟩
    𝐴𝑝𝑝 t ⟦ σ ⊕ s ⟦ σ ⟧ ⟧
      ≡⟨ ap (𝐴𝑝𝑝 t ⟦_⟧) (lem σ (s ⟦ σ ⟧) ⁻¹) ⟩
    𝐴𝑝𝑝 t ⟦ (σ ⊚ π ⊕ 𝑧) ⊚ (𝒾𝒹 Γ ⊕ s ⟦ σ ⟧) ⟧
      ≡⟨ ⟦⟧⟦⟧ (𝐴𝑝𝑝 t) (σ ⊚ π ⊕ 𝑧) (𝒾𝒹 Γ ⊕ s ⟦ σ ⟧) ⁻¹ ⟩
    𝐴𝑝𝑝 t ⟦ σ ⊚ π ⊕ 𝑧 ⟧ ⟦ 𝒾𝒹 Γ ⊕ s ⟦ σ ⟧ ⟧
      ≡⟨ ap _⟦ 𝒾𝒹 Γ ⊕ s ⟦ σ ⟧ ⟧ (𝐴𝑝𝑝⟦⟧ t σ ⁻¹) ⟩
    𝐴𝑝𝑝 (t ⟦ σ ⟧) ⟦ 𝒾𝒹 Γ ⊕ s ⟦ σ ⟧ ⟧
      ≡⟨ 𝑎𝑝𝑝𝐴𝑝𝑝 (t ⟦ σ ⟧) (s ⟦ σ ⟧) ⁻¹ ⟩
    𝑎𝑝𝑝 (t ⟦ σ ⟧) (s ⟦ σ ⟧)
      ∎

  -- A transport lemma

  transp𝐴𝑝𝑝 : {Γ Δ : ctx} {A B : ty} (a : Γ ≡ Δ) (t : tm Γ (A ⇛ B)) →
    transport (λ i → tm (a i ⊹ A) B) (𝐴𝑝𝑝 t) ≡ 𝐴𝑝𝑝 (transport (λ i → tm (a i) (A ⇛ B)) t)
  transp𝐴𝑝𝑝 {A = A} {B} a t =
    J (λ Δ a → transport (λ i → tm (a i ⊹ A) B) (𝐴𝑝𝑝 t)
      ≡ 𝐴𝑝𝑝 (transport (λ i → tm (a i) (A ⇛ B)) t))
      (transportRefl (𝐴𝑝𝑝 t) ∙ ap 𝐴𝑝𝑝 (transportRefl t ⁻¹)) a

-- Next we define what it means for a CF to prserve CCC structures

record CCCPreserving {𝒞 : Contextual ℓ₁ ℓ₂} {𝒟 : Contextual ℓ₃ ℓ₄}
       ⦃ 𝒞CCC : CCC 𝒞 ⦄ ⦃ 𝒟CCC : CCC 𝒟 ⦄ (F : ContextualFunctor 𝒞 𝒟)
       : Type (ℓ₁ ⊔ ℓ₂ ⊔ ℓ₃ ⊔ ℓ₄) where

  private
    module C = Contextual 𝒞
    module D = Contextual 𝒟
    module Cc = CCC 𝒞CCC
    module Dc = CCC 𝒟CCC

  open ContextualFunctor F

  -- We only need to stipulate that it preserves the categorical 𝐴𝑝𝑝
  -- Preservation of Λ and 𝑎𝑝𝑝 follow as corollaries

  field
    pres-⇛ : (A B : C.ty) → CF-ty (A Cc.⇛ B) ≡ CF-ty A Dc.⇛ CF-ty B
    pres-𝐴𝑝𝑝 : {Γ : C.ctx} {A B : C.ty} (t : C.tm Γ (A Cc.⇛ B)) →
      CF-tm (Cc.𝐴𝑝𝑝 t) ≡ Dc.𝐴𝑝𝑝 (transport (λ i → D.tm (CF-ctx Γ) (pres-⇛ A B i)) (CF-tm t))

  pres-Λ : {Γ : C.ctx} {A B : C.ty} (t : C.tm (Γ ⊹ A) B) →
    PathP (λ i → D.tm (CF-ctx Γ) (pres-⇛ A B i)) (CF-tm (Cc.Λ t)) (Dc.Λ (CF-tm t))
  pres-Λ {Γ} {A} {B} t =
    toPathP
      (transport (λ i → D.tm (CF-ctx Γ) (pres-⇛ A B i)) (CF-tm (Cc.Λ t))
        ≡⟨ Dc.𝑎𝑝𝑝η (transport (λ i → D.tm (CF-ctx Γ) (pres-⇛ A B i)) (CF-tm (Cc.Λ t))) ⟩
      Dc.Λ (Dc.𝐴𝑝𝑝 (transport (λ i → D.tm (CF-ctx Γ) (pres-⇛ A B i)) (CF-tm (Cc.Λ t))))
        ≡⟨ ap Dc.Λ (pres-𝐴𝑝𝑝 (Cc.Λ t) ⁻¹) ⟩
      Dc.Λ (CF-tm (Cc.𝐴𝑝𝑝 (Cc.Λ t)))
        ≡⟨ (λ i → Dc.Λ (CF-tm (Cc.𝐴𝑝𝑝β t i))) ⟩
      Dc.Λ (CF-tm t)
        ∎)

  pres-𝑎𝑝𝑝 : {Γ : C.ctx} {A B : C.ty} (t : C.tm Γ (A Cc.⇛ B)) (s : C.tm Γ A) →
    CF-tm (Cc.𝑎𝑝𝑝 t s) ≡
    Dc.𝑎𝑝𝑝 (transport (λ i → D.tm (CF-ctx Γ) (pres-⇛ A B i)) (CF-tm t)) (CF-tm s)
  pres-𝑎𝑝𝑝 {Γ} {A} {B} t s =
    CF-tm (Cc.𝑎𝑝𝑝 t s)
      ≡⟨ ap CF-tm (Cc.𝑎𝑝𝑝𝐴𝑝𝑝 t s) ⟩
    CF-tm (Cc.𝐴𝑝𝑝 t C.⟦ C.𝒾𝒹 Γ ⊕ s ⟧)
      ≡⟨ CF-sub (Cc.𝐴𝑝𝑝 t) (C.𝒾𝒹 Γ ⊕ s) ⟩
    CF-tm (Cc.𝐴𝑝𝑝 t) D.⟦ CF-tms (C.𝒾𝒹 Γ) ⊕ CF-tm s ⟧
      ≡⟨ (λ i → pres-𝐴𝑝𝑝 t i D.⟦ CF-id i ⊕  CF-tm s ⟧) ⟩
    Dc.𝐴𝑝𝑝 (transport (λ i → D.tm (CF-ctx Γ) (pres-⇛ A B i)) (CF-tm t))
      D.⟦ D.𝒾𝒹 (map𝐶𝑡𝑥 CF-ty Γ) ⊕ CF-tm s ⟧
      ≡⟨ Dc.𝑎𝑝𝑝𝐴𝑝𝑝 (transport (λ i → D.tm (CF-ctx Γ) (pres-⇛ A B i)) (CF-tm t)) (CF-tm s) ⁻¹ ⟩
    Dc.𝑎𝑝𝑝 (transport (λ i → D.tm (CF-ctx Γ) (pres-⇛ A B i)) (CF-tm t)) (CF-tm s)
      ∎

-- We define what it means for a CCC to be initial

module _ (𝒞 : Contextual ℓ ℓ) ⦃ 𝒞CCC : CCC 𝒞 ⦄ (base₁ : Char → Contextual.ty 𝒞) where
  open Contextual
  open ContextualFunctor

  record InitialInstance (𝒟 : Contextual ℓ₁ ℓ₂) ⦃ 𝒟CCC : CCC 𝒟 ⦄ (base₂ : Char → ty 𝒟)
                         : Type (ℓ ⊔ ℓ₁ ⊔ ℓ₂) where
    constructor initIn

    BasePreserving : ContextualFunctor 𝒞 𝒟 → Type ℓ₁
    BasePreserving F = (c : Char) → CF-ty F (base₁ c) ≡ base₂ c

    field
      elim : ContextualFunctor 𝒞 𝒟
      ccc-pres : CCCPreserving elim
      base-pres : BasePreserving elim
      UP : (F : ContextualFunctor 𝒞 𝒟) → CCCPreserving F → BasePreserving F → F ≡ elim

  InitialCCC = ∀ {ℓ₁} {ℓ₂} (𝒟 : Contextual ℓ₁ ℓ₂) ⦃ 𝒟CCC : CCC 𝒟 ⦄ (base₂ : Char → ty 𝒟) →
    InitialInstance 𝒟 base₂

-- Proof that the composition of CCC preserving CFs is CCC preserving
-- Welcome to the ninth circle of transport hell

module _ {𝒞 : Contextual ℓ₁ ℓ₂} {𝒟 : Contextual ℓ₃ ℓ₄} {ℰ : Contextual ℓ₅ ℓ₆}
         ⦃ 𝒞CCC : CCC 𝒞 ⦄ ⦃ 𝒟CCC : CCC 𝒟 ⦄ ⦃ ℰCCC : CCC ℰ ⦄
         {G : ContextualFunctor 𝒟 ℰ} {F : ContextualFunctor 𝒞 𝒟} where
  open ContextualFunctor
  open CCCPreserving
  open CCC

  private
    module C = Contextual 𝒞
    module D = Contextual 𝒟
    module E = Contextual ℰ
    module Cc = CCC 𝒞CCC
    module Dc = CCC 𝒟CCC
    module Ec = CCC ℰCCC

  ∘CF-CCCPres : CCCPreserving G → CCCPreserving F → CCCPreserving (G ∘CF F)
  pres-⇛ (∘CF-CCCPres p₁ p₂) A B =
    ap (CF-ty G) (pres-⇛ p₂ A B) ∙ (pres-⇛ p₁ (CF-ty F A) (CF-ty F B))
  pres-𝐴𝑝𝑝 (∘CF-CCCPres p₁ p₂) {Γ} {A} {B} t =
    transport (λ i → E.tm (map𝐶𝑡𝑥comp (CF-ty G) (CF-ty F) (Γ ⊹ A) i) (CF-ty G (CF-ty F B)))
      (CF-tm G (CF-tm F (Cc.𝐴𝑝𝑝 t)))
      ≡⟨ ap (transport (λ i → E.tm (map𝐶𝑡𝑥comp (CF-ty G) (CF-ty F) (Γ ⊹ A) i)
        (CF-ty G (CF-ty F B)))) lem ⟩
    transport (λ i → E.tm (map𝐶𝑡𝑥comp (CF-ty G) (CF-ty F) (Γ ⊹ A) i) (CF-ty G (CF-ty F B)))
      (Ec.𝐴𝑝𝑝 (transport (λ i → E.tm (CF-ctx G (CF-ctx F Γ)) ((ap (CF-ty G) (pres-⇛ p₂ A B)
        ∙ pres-⇛ p₁ (CF-ty F A) (CF-ty F B)) i)) (CF-tm G (CF-tm F t))))
      ≡⟨ transp𝐴𝑝𝑝 ℰCCC (map𝐶𝑡𝑥comp (CF-ty G) (CF-ty F) Γ) (transport (λ i → E.tm
        (CF-ctx G (CF-ctx F Γ)) ((ap (CF-ty G) (pres-⇛ p₂ A B)
        ∙ pres-⇛ p₁ (CF-ty F A) (CF-ty F B)) i)) (CF-tm G (CF-tm F t))) ⟩
    Ec.𝐴𝑝𝑝 (transport
      (λ i → E.tm (map𝐶𝑡𝑥comp (CF-ty G) (CF-ty F) Γ i)
        (CF-ty G (CF-ty F A) Ec.⇛ CF-ty G (CF-ty F B)))
      (transport
        (λ i → E.tm (CF-ctx G (CF-ctx F Γ)) ((ap (CF-ty G) (pres-⇛ p₂ A B)
          ∙ pres-⇛ p₁ (CF-ty F A) (CF-ty F B)) i))
        (CF-tm G (CF-tm F t))))
      ≡⟨ ap Ec.𝐴𝑝𝑝 (transport-tm {tm = E.tm} refl (ap (CF-ty G) (pres-⇛ p₂ A B)
        ∙ pres-⇛ p₁ (CF-ty F A) (CF-ty F B)) (map𝐶𝑡𝑥comp (CF-ty G) (CF-ty F) Γ) refl
        (CF-tm G (CF-tm F t))) ⟩
    Ec.𝐴𝑝𝑝 (transport (λ i → E.tm
      ((refl ∙ map𝐶𝑡𝑥comp (CF-ty G) (CF-ty F) Γ) i)
      (((ap (CF-ty G) (pres-⇛ p₂ A B) ∙ pres-⇛ p₁ (CF-ty F A) (CF-ty F B)) ∙ refl) i))
      (CF-tm G (CF-tm F t)))
      ≡⟨ (λ j → Ec.𝐴𝑝𝑝 (transport (λ i → E.tm
        (rUnit (lUnit (map𝐶𝑡𝑥comp (CF-ty G) (CF-ty F) Γ) (~ j)) j i)
        (lUnit (rUnit (ap (CF-ty G) (pres-⇛ p₂ A B)
          ∙ pres-⇛ p₁ (CF-ty F A) (CF-ty F B)) (~ j)) j i))
        (CF-tm G (CF-tm F t)))) ⟩
    Ec.𝐴𝑝𝑝 (transport (λ i → E.tm
      ((map𝐶𝑡𝑥comp (CF-ty G) (CF-ty F) Γ ∙ refl) i)
      ((refl ∙ (ap (CF-ty G) (pres-⇛ p₂ A B) ∙ pres-⇛ p₁ (CF-ty F A) (CF-ty F B))) i))
      (CF-tm G (CF-tm F t)))
      ≡⟨ ap Ec.𝐴𝑝𝑝 (transport-tm {tm = E.tm} (map𝐶𝑡𝑥comp (CF-ty G) (CF-ty F) Γ) refl
        refl (ap (CF-ty G) (pres-⇛ p₂ A B) ∙ pres-⇛ p₁ (CF-ty F A) (CF-ty F B))
        (CF-tm G (CF-tm F t)) ⁻¹) ⟩
    Ec.𝐴𝑝𝑝 (transport (λ i → E.tm (map𝐶𝑡𝑥 (CF-ty G ∘ CF-ty F) Γ)
      ((ap (CF-ty G) (pres-⇛ p₂ A B) ∙ pres-⇛ p₁ (CF-ty F A) (CF-ty F B)) i))
      (transport (λ i → E.tm (map𝐶𝑡𝑥comp (CF-ty G) (CF-ty F) Γ i) (CF-ty G (CF-ty F (A Cc.⇛ B))))
        (CF-tm G (CF-tm F t))))
      ∎ where
    lem : CF-tm G (CF-tm F (Cc.𝐴𝑝𝑝 t))
      ≡ Ec.𝐴𝑝𝑝 (transport (λ i → E.tm (CF-ctx G (CF-ctx F Γ)) ((ap (CF-ty G) (pres-⇛ p₂ A B)
        ∙ pres-⇛ p₁ (CF-ty F A) (CF-ty F B)) i)) (CF-tm G (CF-tm F t)))
    lem =
      CF-tm G (CF-tm F (Cc.𝐴𝑝𝑝 t))
        ≡⟨ ap (CF-tm G) (pres-𝐴𝑝𝑝 p₂ t) ⟩
      CF-tm G (Dc.𝐴𝑝𝑝 (transport (λ i → D.tm (CF-ctx F Γ) (pres-⇛ p₂ A B i)) (CF-tm F t)))
        ≡⟨ pres-𝐴𝑝𝑝 p₁ (transport (λ i → D.tm (CF-ctx F Γ) (pres-⇛ p₂ A B i)) (CF-tm F t)) ⟩
      Ec.𝐴𝑝𝑝 (transport (λ i → E.tm (CF-ctx G (CF-ctx F Γ)) (pres-⇛ p₁ (CF-ty F A) (CF-ty F B) i))
        (CF-tm G (transport (λ i → D.tm (CF-ctx F Γ) (pres-⇛ p₂ A B i)) (CF-tm F t))))
        ≡⟨ (λ i → Ec.𝐴𝑝𝑝 (transport (λ i → E.tm (CF-ctx G (CF-ctx F Γ)) (pres-⇛ p₁ (CF-ty F A)
          (CF-ty F B) i)) (transpCF-tm G (pres-⇛ p₂ A B) (CF-tm F t) (~ i)))) ⟩
      Ec.𝐴𝑝𝑝 (transport (λ i → E.tm (CF-ctx G (CF-ctx F Γ)) (pres-⇛ p₁ (CF-ty F A) (CF-ty F B) i))
        (transport (λ i → E.tm (CF-ctx G (map𝐶𝑡𝑥 (CF-ty F) Γ)) (CF-ty G (pres-⇛ p₂ A B i)))
          (CF-tm G (CF-tm F t))))
        ≡⟨ ap Ec.𝐴𝑝𝑝 (transport-tm {tm = E.tm} refl (ap (CF-ty G) (pres-⇛ p₂ A B))
          refl (pres-⇛ p₁ (CF-ty F A) (CF-ty F B)) (CF-tm G (CF-tm F t))) ⟩
      Ec.𝐴𝑝𝑝 (transport (λ i → E.tm ((refl {x = CF-ctx G (CF-ctx F Γ)} ∙ refl) i)
        ((ap (CF-ty G) (pres-⇛ p₂ A B) ∙ pres-⇛ p₁ (CF-ty F A) (CF-ty F B)) i))
        (CF-tm G (CF-tm F t)))
        ≡⟨ (λ j → Ec.𝐴𝑝𝑝 (transport (λ i → E.tm (rUnit (refl {x = CF-ctx G (CF-ctx F Γ)}) (~ j) i)
          ((ap (CF-ty G) (pres-⇛ p₂ A B) ∙ pres-⇛ p₁ (CF-ty F A) (CF-ty F B)) i))
          (CF-tm G (CF-tm F t)))) ⟩
      Ec.𝐴𝑝𝑝 (transport (λ i → E.tm (CF-ctx G (CF-ctx F Γ)) ((ap (CF-ty G) (pres-⇛ p₂ A B)
        ∙ pres-⇛ p₁ (CF-ty F A) (CF-ty F B)) i)) (CF-tm G (CF-tm F t)))
        ∎
\end{code}
