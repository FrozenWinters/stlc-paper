\begin{code}[hide]
{-# OPTIONS --cubical #-}

module contextual where

open import lists public

open import Cubical.Categories.Category
open import Cubical.Categories.Functor
open import Agda.Builtin.Char public

private
  variable
    ℓ₁ ℓ₂ ℓ₃ ℓ₄ ℓ₅ ℓ₆ : Level

-- This new definition of a contextual category arose as a way to de-boilerplate the code;
-- it is the most natural variation of the definition to use in an implementation.

record Contextual (ℓ₁ ℓ₂ : Level) : Type (lsuc (ℓ₁ ⊔ ℓ₂))

𝑎𝑚𝑏Cat : Contextual ℓ₁ ℓ₂ → Precategory ℓ₁ (ℓ₁ ⊔ ℓ₂)
isCat𝐴𝑚𝑏Cat : (𝒞 : Contextual ℓ₁ ℓ₂) → isCategory (𝑎𝑚𝑏Cat 𝒞)
\end{code}

\subsection{Contextual Categories}

The data of a category consists of a type of objects, h-sets of morphisms, and
distinguished identities. This data is required to satisfy left and right
identity laws as well as associativity.

Renamings, as studied above, form a category of a particular kind; the objects
of this category are lists of $\mathsf{ty}$, and the morphisms are indexed lists
of $\mathit{Var}~\mathsf{ty}$, moreover, the composition of morphisms is derived
from a more primitive substitution operation $t~[~\sigma~]$.

Instead of defining a \emph{Contextual Category} to be a category which happens
to be of this form, we rather define its structure to consist of the more
primitive notions of types, terms, and substitution \emph{(action)}. From this,
the notions of contexts, substitutions \emph{(objects)}, and composition are to be
derived. The right identity law and associativity, consequently, are derived
from more primitive laws involving substitution instead of composition.

\begin{AgdaMultiCode}
\begin{code}
record Contextual ℓ₁ ℓ₂ where

  field
    ty : Type ℓ₁

  ctx = 𝐶𝑡𝑥 ty

  field
    tm : ctx → ty → Type ℓ₂

  tms = 𝑇𝑚𝑠 tm

  field
    _⟦_⟧ : {Γ Δ : ctx} {A : ty} → tm Δ A → tms Γ Δ → tm Γ A

  _⊚_ : {Γ Δ Σ : ctx} → tms Δ Σ → tms Γ Δ → tms Γ Σ
  σ ⊚ τ = map𝑇𝑚𝑠 _⟦ τ ⟧ σ
\end{code}
\begin{code}[hide]
  infixl 30 _⟦_⟧
  infixl 20 _⊚_
\end{code}
\begin{code}

  field
    𝒾𝒹 : (Γ : ctx) → tms Γ Γ
    𝒾𝒹L : {Γ Δ : ctx} (σ : tms Γ Δ) → 𝒾𝒹 Δ ⊚ σ ≡ σ
    𝒾𝒹⟦⟧ : {Γ : ctx} {A : ty} (t : tm Γ A) → t ⟦ 𝒾𝒹 Γ ⟧ ≡ t
    ⟦⟧⟦⟧ : {Γ Δ Σ : ctx} {A : ty} (t : tm Σ A) (σ : tms Δ Σ) (τ : tms Γ Δ) →
      t ⟦ σ ⟧ ⟦ τ ⟧ ≡ t ⟦ σ ⊚ τ ⟧
    isSetTm : {Γ : ctx} {A : ty} → isSet (tm Γ A)

  𝒾𝒹R : {Γ Δ : ctx} (σ : tms Γ Δ) → σ ⊚ 𝒾𝒹 Γ ≡ σ
  𝒾𝒹R ! = refl
  𝒾𝒹R (σ ⊕ t) i = 𝒾𝒹R σ i ⊕ 𝒾𝒹⟦⟧ t i

  ⊚Assoc : {Γ Δ Σ Ω : ctx} (σ : tms Σ Ω) (τ : tms Δ Σ) (μ : tms Γ Δ) →
    (σ ⊚ τ) ⊚ μ ≡ σ ⊚ (τ ⊚ μ)
  ⊚Assoc ! τ μ = refl
  ⊚Assoc (σ ⊕ t) τ μ i = ⊚Assoc σ τ μ i ⊕ ⟦⟧⟦⟧ t τ μ i
\end{code}
\end{AgdaMultiCode}

The eight fields of data above comprise the totality of what is needed to
construct a contextual category. In the codebase, however, the record definition
$\mathsf{Contextual}$ continues for quite some length with a fair number of
derived constructions that apply to all contextual categories.

In the first instance, we collect the data above in order to show that every
contextual category induces an \emph{Ambient Category} of contexts and
substitutions. In the code, this is given by the derived field
$\mathsf{ambCat}$.

\begin{code}[hide]
  private
    module P = 𝑇𝑚𝑠Path tm isSetTm

  isSetTms = P.isSet𝑇𝑚𝑠

  -- Every contextual category has an ambient category of contexts and terms

  open Precategory hiding (_∘_)

  ambCat : Precategory ℓ₁ (ℓ₁ ⊔ ℓ₂)
  ob ambCat = ctx
  Hom[_,_] ambCat Γ Δ = tms Γ Δ
  Precategory.id ambCat Γ = 𝒾𝒹 Γ
  _⋆_ ambCat τ σ = σ ⊚ τ
  ⋆IdL ambCat = 𝒾𝒹R
  ⋆IdR ambCat = 𝒾𝒹L
  ⋆Assoc ambCat μ τ σ = ⊚Assoc σ τ μ ⁻¹

  isCatAmbCat : isCategory ambCat
  isSetHom isCatAmbCat = isSetTms

  -- ∅ is automatically the terminal object with unique morphism !

  !η : {Γ : ctx} (σ : tms Γ ∅) → ! ≡ σ
  !η ! = refl

  -- Contextual categories automatically have products
\end{code}

In the second instance, applying \emph{first} and \emph{rest} to the identity
substitution yields two fundemental context projections:
\begin{code}
  𝑧 : {Γ : ctx} {A : ty} → tm (Γ ⊹ A) A
  𝑧 {Γ} {A} = 𝑧𝑇𝑚𝑠 (𝒾𝒹 (Γ ⊹ A))

  π : {Γ : ctx} {A : ty} → tms (Γ ⊹ A) Γ
  π {Γ} {A} = π𝑇𝑚𝑠 (𝒾𝒹 (Γ ⊹ A))
\end{code}
\begin{code}[hide]
  𝒾𝒹η : {Γ : ctx} {A : ty} → π ⊕ 𝑧 ≡ 𝒾𝒹 (Γ ⊹ A)
  𝒾𝒹η {Γ} {A} = π𝑧η𝑇𝑚𝑠 (𝒾𝒹 (Γ ⊹ A))

  π𝑧η : {Γ Δ : ctx} {A : ty} (σ : tms Γ (Δ ⊹ A)) →
    (π ⊚ σ) ⊕ (𝑧 ⟦ σ ⟧) ≡ σ
  π𝑧η {Γ} {Δ} {A} σ =
    π ⊚ σ ⊕ 𝑧 ⟦ σ ⟧
      ≡⟨ ap (_⊚ σ) 𝒾𝒹η ⟩
    𝒾𝒹 (Δ ⊹ A) ⊚ σ
      ≡⟨ 𝒾𝒹L σ ⟩
    σ
      ∎

  πβ : {Γ Δ : ctx} {A : ty} (σ : tms Γ (Δ ⊹ A)) →
    π ⊚ σ ≡ π𝑇𝑚𝑠 σ
  πβ σ = ap π𝑇𝑚𝑠 (π𝑧η σ)

  𝑧β : {Γ Δ : ctx} {A : ty} (σ : tms Γ (Δ ⊹ A)) →
    𝑧 ⟦ σ ⟧ ≡ 𝑧𝑇𝑚𝑠 σ
  𝑧β σ = ap 𝑧𝑇𝑚𝑠 (π𝑧η σ)
\end{code}
\noindent
This allows us to easily derive a weakening operation on terms:
\begin{code}
  W₁tm : {Γ : ctx} (A : ty) {B : ty} → tm Γ B → tm (Γ ⊹ A) B
  W₁tm A t = t ⟦ π ⟧
\end{code}
\begin{code}[hide]
  W₁tms : {Γ Δ : ctx} (A : ty) → tms Γ Δ → tms (Γ ⊹ A) Δ
  W₁tms A σ = σ ⊚ π

  W₂tms : {Γ Δ : ctx} (A : ty) → tms Γ Δ → tms (Γ ⊹ A) (Δ ⊹ A)
  W₂tms A σ = W₁tms A σ ⊕ 𝑧
\end{code}
\noindent
And we can moreover specify how substitution acts on weakened terms:
\begin{code}
  W₁lem1 : {Γ Δ : ctx} {A B : ty} (t : tm Δ B) (σ : tms Γ Δ) (s : tm Γ A) →
    W₁tm A t ⟦ σ ⊕ s ⟧ ≡ t ⟦ σ ⟧
\end{code}
\begin{code}[hide]
  W₁lem1 t σ s =
    t ⟦ π ⟧ ⟦ σ ⊕ s ⟧
      ≡⟨ ⟦⟧⟦⟧ t π (σ ⊕ s) ⟩
    t ⟦ π ⊚ (σ ⊕ s) ⟧
      ≡⟨ ap (t ⟦_⟧) (πβ (σ ⊕ s)) ⟩
    t ⟦ σ ⟧
      ∎

  W₁lem2 : {Γ Δ Σ : ctx} {A : ty} (σ : tms Δ Σ) (τ : tms Γ Δ) (t : tm Γ A) →
    W₁tms A σ ⊚ (τ ⊕ t) ≡ σ ⊚ τ
  W₁lem2 ! τ t = refl
  W₁lem2 (σ ⊕ s) τ t i = W₁lem2 σ τ t i ⊕ W₁lem1 s τ t i
\end{code}
However, this merely scratches the surface of what it means for the identity
substitution to be composed of terms! We begin by defining variables and
renamings internal to a contextual category:
\begin{code}
  IntVar = 𝑉𝑎𝑟 ty
  IntRen = 𝑅𝑒𝑛 ty
\end{code}
From here, we define indexing functions that allow us to use an internal
variable to pick out a component of the identity function. We will also refer to
these components as \emph{Variables}.
\begin{code}
  makeVar : {Γ : ctx} {A : ty} → IntVar Γ A → tm Γ A
  makeVar {Γ} = derive (𝒾𝒹 Γ)

  makeRen : {Γ Δ : ctx} → IntRen Γ Δ → tms Γ Δ
  makeRen {Γ} = derive𝑅𝑒𝑛 (𝒾𝒹 Γ)
\end{code}
\begin{code}[hide]
  ρεν : Contextual ℓ₁ ℓ₁
  ty ρεν = ty
  tm ρεν = IntVar
  _⟦_⟧ ρεν = _[_]𝑅
  𝒾𝒹 ρεν Γ = id𝑅𝑒𝑛 Γ
  𝒾𝒹L ρεν = ∘𝑅𝑒𝑛IdL
  𝒾𝒹⟦⟧ ρεν = [id]𝑅𝑒𝑛
  ⟦⟧⟦⟧ ρεν = [][]𝑅𝑒𝑛
  isSetTm ρεν = 𝑉𝑎𝑟Path.isSet𝑉𝑎𝑟

  REN : Precategory ℓ₁ ℓ₁
  REN = 𝑎𝑚𝑏Cat ρεν

  instance
    isCat𝑅𝑒𝑛 : isCategory REN
    isCat𝑅𝑒𝑛 = isCat𝐴𝑚𝑏Cat ρεν

  derive⟦⟧ : {Γ Δ Σ : ctx} {A : ty} (v : IntVar Σ A) (σ : tms Δ Σ) (τ : tms Γ Δ) →
    derive σ v ⟦ τ ⟧ ≡ derive (σ ⊚ τ) v
  derive⟦⟧ 𝑧𝑣 σ τ =
    𝑧𝑇𝑚𝑠 σ ⟦ τ ⟧
      ≡⟨ ap _⟦ τ ⟧ (𝑧β σ ⁻¹) ⟩
    𝑧 ⟦ σ ⟧ ⟦ τ ⟧
      ≡⟨ ⟦⟧⟦⟧ 𝑧 σ τ ⟩
    𝑧 ⟦ σ ⊚ τ ⟧
      ≡⟨ 𝑧β (σ ⊚ τ) ⟩
    𝑧𝑇𝑚𝑠 (σ ⊚ τ)
      ∎
  derive⟦⟧ (𝑠𝑣 v) σ τ =
    derive (π𝑇𝑚𝑠 σ) v ⟦ τ ⟧
      ≡⟨ (λ i → derive (πβ σ (~ i)) v ⟦ τ ⟧) ⟩
    derive (π ⊚ σ) v ⟦ τ ⟧
      ≡⟨ ap _⟦ τ ⟧ (derive⟦⟧ v π σ ⁻¹) ⟩
    derive π v ⟦ σ ⟧ ⟦ τ ⟧
      ≡⟨ ⟦⟧⟦⟧ (derive π v) σ τ ⟩
    derive π v ⟦ σ ⊚ τ ⟧
      ≡⟨ derive⟦⟧ v π (σ ⊚ τ) ⟩
    derive (π ⊚ (σ ⊚ τ)) v
      ≡⟨ (λ i → derive (πβ (σ ⊚ τ) i) v) ⟩
    derive (π𝑇𝑚𝑠 (σ ⊚ τ)) v
      ∎
\end{code}
\noindent
Through a general lemma on indexing the components of a composition of
substitutions, we derive a result that says that substituting into a variable
serves to pick out a component of the substitution:
\begin{code}
  var⟦⟧ : {Γ Δ : ctx} {A : ty} (v : IntVar Δ A) (σ : tms Γ Δ) →
    makeVar v ⟦ σ ⟧ ≡ derive σ v
\end{code}
At this point, we remark that we could have alternatively defined a contextual
category to consist of the data of its variables in place of the data of its
identity substitutions. In this case, the identity substitution would be derived
by grouping variables, and the left identity law would be derived from a
primitive instance of the above theorem. \emph{(While there is a case to be made
that this alternative approach is more aesthetic, the resulting theory is
logically equivalent to ours and is admits similar ease of use.)}
\begin{code}[hide]
  var⟦⟧ {Γ} {Δ} v σ =
    derive (𝒾𝒹 Δ) v ⟦ σ ⟧
      ≡⟨ derive⟦⟧ v (𝒾𝒹 Δ) σ ⟩
    derive (𝒾𝒹 Δ ⊚ σ) v
      ≡⟨ (λ i → derive (𝒾𝒹L σ i) v) ⟩
    derive σ v
      ∎
\end{code}

The fun, however, does not stop here! We are additionally able to derive a
relationship that different variables satisfy:
\begin{code}
  make𝑠𝑣 : {Γ : ctx} {A B : ty} (v : IntVar Γ B) →
    makeVar (𝑠𝑣 v) ≡ W₁tm A (makeVar v)
\end{code}
\begin{code}[hide]
  make𝑠𝑣 v = var⟦⟧ v π ⁻¹

  makeW₁ : {Γ Δ : ctx} {A : ty} (σ : IntRen Γ Δ) →
    makeRen (W₁𝑅𝑒𝑛 A σ) ≡ W₁tms A (makeRen σ)
  makeW₁ ! = refl
  makeW₁ (σ ⊕ v) i = makeW₁ σ i ⊕ make𝑠𝑣 v i

  deriveW₁ : {Γ Δ Σ : ctx} {A : ty} (σ : tms Γ Δ) (t : tm Γ A) (v : IntRen Δ Σ) →
    derive𝑅𝑒𝑛 (σ ⊕ t) (W₁𝑅𝑒𝑛 A v) ≡ derive𝑅𝑒𝑛 σ v
  deriveW₁ σ t ! = refl
  deriveW₁ σ t (τ ⊕ v) i = deriveW₁ σ t τ i ⊕ derive σ v

  W₁⟦⟧ : {Γ Δ : ctx} {A B : ty} (v : IntVar Δ B) (σ : tms Γ Δ) (t : tm Γ A) →
    makeVar (𝑠𝑣 v) ⟦ σ ⊕ t ⟧ ≡ makeVar v ⟦ σ ⟧
  W₁⟦⟧ v σ t =
    makeVar (𝑠𝑣 v) ⟦ σ ⊕ t ⟧
      ≡⟨ ap _⟦ σ ⊕ t ⟧ (make𝑠𝑣 v) ⟩
    W₁tm _ (makeVar v) ⟦ σ ⊕ t ⟧
      ≡⟨ W₁lem1 (makeVar v) σ t ⟩
    makeVar v ⟦ σ ⟧
      ∎
\end{code}
\noindent
From the above, we obtain a functoriality property of $\mathsf{makeVar}$:
\begin{code}
  make[]𝑅 : {Γ Δ : ctx} {A : ty} (v : IntVar Δ A) (σ : IntRen Γ Δ) →
    makeVar (v [ σ ]𝑅) ≡ makeVar v ⟦ makeRen σ ⟧
\end{code}
\begin{code}[hide]
  make[]𝑅 𝑧𝑣 (σ ⊕ t) = 𝑧β (makeRen (σ ⊕ t)) ⁻¹
  make[]𝑅 (𝑠𝑣 v) (σ ⊕ t) =
    makeVar (v [ σ ]𝑅)
      ≡⟨ make[]𝑅 v σ ⟩
    makeVar v ⟦ makeRen σ ⟧
      ≡⟨ W₁⟦⟧ v (makeRen σ) (makeVar t) ⁻¹ ⟩
    makeVar (𝑠𝑣 v) ⟦ makeRen (σ ⊕ t) ⟧
      ∎

  make∘𝑅𝑒𝑛 : {Γ Δ Σ : ctx} (σ : IntRen Δ Σ) (τ : IntRen Γ Δ) →
    makeRen (σ ∘𝑅𝑒𝑛 τ) ≡ makeRen σ ⊚ makeRen τ
  make∘𝑅𝑒𝑛 ! τ = refl
  make∘𝑅𝑒𝑛 (σ ⊕ v) τ i = make∘𝑅𝑒𝑛 σ τ i ⊕ make[]𝑅 v τ i

  derive𝒾𝒹 : {Γ Δ : ctx} (σ : tms Γ Δ) →
    derive𝑅𝑒𝑛 σ (id𝑅𝑒𝑛 Δ) ≡ σ
  derive𝒾𝒹 ! = refl
  derive𝒾𝒹 {Γ} {Δ ⊹ A} (σ ⊕ t) =
    derive𝑅𝑒𝑛 (σ ⊕ t) (W₁𝑅𝑒𝑛 A (id𝑅𝑒𝑛 Δ)) ⊕ t
      ≡⟨ ap (_⊕ t) (deriveW₁ σ t (id𝑅𝑒𝑛 Δ)) ⟩
    derive𝑅𝑒𝑛 σ (id𝑅𝑒𝑛 Δ) ⊕ t
      ≡⟨ ap (_⊕ t) (derive𝒾𝒹 σ) ⟩
    σ ⊕ t
      ∎
\end{code}
\noindent
And, separately, we exhibit a way of assembling the identity from variables:
\begin{code}
  𝒾𝒹η₂ : {Γ : ctx} → makeRen (id𝑅𝑒𝑛 Γ) ≡ 𝒾𝒹 Γ
\end{code}
As we will soon see, the above two laws assert the existence of a
\emph{Contextual Functor} to any contextual category from its internal
contextual category of renamings.
\iffalse
The above has been a tour of some of the constructions that we can make in any
contextual category. Every single one of these construction was created from
necessity and is used somewhere in the codebase. As one can see, contextual
categories admit a rich theory of variables, allowing us to recover much of the
familiar structure of type theory.
\fi
\begin{code}[hide]
  𝒾𝒹η₂ {Γ} = derive𝒾𝒹 (𝒾𝒹 Γ)

  πη : {Γ : ctx} {A : ty} → makeRen (W₁𝑅𝑒𝑛 A (id𝑅𝑒𝑛 Γ)) ≡ π
  πη {Γ} {A} =
    makeRen (W₁𝑅𝑒𝑛 A (id𝑅𝑒𝑛 Γ))
      ≡⟨ makeW₁ (id𝑅𝑒𝑛 Γ) ⟩
    W₁tms A (makeRen (id𝑅𝑒𝑛 Γ))
      ≡⟨ ap (W₁tms A) 𝒾𝒹η₂ ⟩
    𝒾𝒹 Γ ⊚ π
      ≡⟨ 𝒾𝒹L π ⟩
    π
      ∎

  -- Some transport lemmas
  transp𝒾𝒹 : {Δ Σ : ctx} (a : Δ ≡ Σ) →
    transport (λ i → tms (a i) (a i)) (𝒾𝒹 Δ) ≡ 𝒾𝒹 Σ
  transp𝒾𝒹 {Δ} {Σ} a =
    J (λ Σ a → transport (λ i → tms (a i) (a i)) (𝒾𝒹 Δ) ≡ 𝒾𝒹 Σ)
      (transportRefl (𝒾𝒹 Δ)) a

  transp⟦⟧ : {Γ₁ Γ₂ Δ₁ Δ₂ : ctx} {A : ty} (a : Γ₁ ≡ Γ₂)
    (b : Δ₁ ≡ Δ₂) (t : tm Δ₁ A) (σ : tms Γ₁ Δ₁) →
    transport (λ i → tm (a i) A) (t ⟦ σ ⟧)
      ≡ transport (λ i → tm (b i) A) t ⟦ transport (λ i → tms (a i) (b i)) σ ⟧
  transp⟦⟧ {Γ₁} {Γ₂} {Δ₁} {Δ₂} {A} a b t σ =
    J (λ Γ₂ a → transport (λ i → tm (a i) A) (t ⟦ σ ⟧)
      ≡ transport (λ i → tm (b i) A) t ⟦ transport (λ i → tms (a i) (b i)) σ ⟧)
      (J (λ Δ₂ b → transport (λ i → tm Γ₁ A) (t ⟦ σ ⟧) ≡
        transport (λ i → tm (b i) A) t ⟦ transport (λ i → tms Γ₁ (b i)) σ ⟧)
        (transportRefl (t ⟦ σ ⟧) ∙ (λ i → transportRefl t (~ i) ⟦ transportRefl σ (~ i) ⟧))
        b) a

𝑎𝑚𝑏Cat 𝒞 = Contextual.ambCat 𝒞
isCat𝐴𝑚𝑏Cat 𝒞 = Contextual.isCatAmbCat 𝒞

module _ (𝒞 : Contextual ℓ₁ ℓ₂) where
  open Contextual 𝒞
  open Functor

  ιREN : Functor REN ambCat
  F-ob ιREN Γ = Γ
  F-hom ιREN σ = makeRen σ
  F-id ιREN = 𝒾𝒹η₂
  F-seq ιREN σ τ = make∘𝑅𝑒𝑛 τ σ

  private
    module R = Contextual ρεν

  makeRenVar : {Γ : ctx} {A : ty} (v : R.IntVar Γ A) → R.makeVar v ≡ v
  makeRenVar 𝑧𝑣 = refl
  makeRenVar {Γ ⊹ A} (𝑠𝑣 v) =
    derive (map𝑇𝑚𝑠 𝑠𝑣 (id𝑅𝑒𝑛 Γ)) v
      ≡⟨ deriveMap 𝑠𝑣 (id𝑅𝑒𝑛 Γ) v ⟩
    𝑠𝑣 (derive (id𝑅𝑒𝑛 Γ) v)
      ≡⟨ ap 𝑠𝑣 (makeRenVar v) ⟩
    𝑠𝑣 v
      ∎
\end{code}

\subsection{Contextual Functors}

A \textbf{contextual functor} between contextual categories $\mathcal{C}$ and
$\mathcal{D}$, briefly, sends types to types, terms to terms, and is
additionally required to preserve the identity and substitution. These notions
extend to be defined on contexts and substitutions, and we are able to show
preservation of composition.

\begin{code}
record ContextualFunctor (𝒞 : Contextual ℓ₁ ℓ₂) (𝒟 : Contextual ℓ₃ ℓ₄)
       : Type (ℓ₁ ⊔ ℓ₂ ⊔ ℓ₃ ⊔ ℓ₄) where
  open Contextual

  private
    module C = Contextual 𝒞
    module D = Contextual 𝒟

  field
    CF-ty : C.ty → D.ty

  CF-ctx : C.ctx → D.ctx
  CF-ctx Γ = map𝐶𝑡𝑥 CF-ty Γ

  field
    CF-tm : {Γ : C.ctx} {A : C.ty} → C.tm Γ A → D.tm (CF-ctx Γ) (CF-ty A)

  CF-tms : {Γ Δ : C.ctx} → C.tms Γ Δ → D.tms (CF-ctx Γ) (CF-ctx Δ)
  CF-tms σ = map𝑇𝑚𝑠₁ CF-tm σ

  field
    CF-id : {Γ : C.ctx} → CF-tms (C.𝒾𝒹 Γ) ≡ D.𝒾𝒹 (CF-ctx Γ)
    CF-sub : {Γ Δ : C.ctx} {A : C.ty} (t : C.tm Δ A) (σ : C.tms Γ Δ) →
      CF-tm (t C.⟦ σ ⟧) ≡ CF-tm t D.⟦ CF-tms σ ⟧

  CF-comp : {Γ Δ Σ : C.ctx} (σ : C.tms Δ Σ) (τ : C.tms Γ Δ) →
    CF-tms (σ C.⊚ τ) ≡ (CF-tms σ) D.⊚ (CF-tms τ)
  CF-comp ! τ = refl
  CF-comp (σ ⊕ t) τ i = CF-comp σ τ i ⊕ CF-sub t τ i
\end{code}
\clearpage

Two new functions are present in the above: $\mathsf{map}\mathit{Ctx}$ and
$\mathsf{map}\mathit{Tms}_\mathit{1}$. Both are essentially still the list
function \emph{map}, but the latter differs from $\mathsf{map}\mathit{Tms}$ in
its type signature. $\mathsf{map}\mathit{Tms}$ takes a function $f : \{A :
\mathsf{ty}\} \to \mathsf{tm}_\mathit{1}~\Gamma_\mathit{1}~A \to
\mathsf{tm}_\mathit{2}~\Gamma_\mathit{2}~A$, where $\mathsf{tm}_\mathit{1}$ and
$\mathsf{tm}_\mathit{2}$ share the same notion of types. We thus have that, if
$\sigma : \mathit{Tms}~\mathsf{tm}_\mathit{1}~\Gamma_\mathit{1}~\Delta$, then
$\mathsf{map}\mathit{Tms}~f~\sigma : \mathit{Tms}~\mathsf{tm}_\mathit{2}
~\Gamma_\mathit{2}~\Delta$. In particular, the codomain context is preserved
definitionally. On the other hand, $\mathsf{map}\mathit{Tms}_\mathit{1}$ more
generally takes a function of type $f : \{A : \mathsf{ty}_\mathit{1}\} \to
\mathsf{tm}_\mathit{1}~\Gamma_\mathit{1}~A \to \mathsf{tm}_\mathit{2}~\Gamma
_\mathit{2}~(P~A)$ where $P : \mathsf{ty}_\mathit{1} \to \mathsf{ty}
_\mathit{2}$. As a consequence, we have that $\mathsf{map}\mathit{Tms}
_\mathit{1}~f~\sigma : \mathit{Tms}~\mathsf{tm}_\mathit{2}~\Gamma_\mathit{2}
~(\mathsf{map}\mathit{Ctx}~P~\Delta)$.

One will of course note that the former function \emph{is not} obtained by
specialising the latter to the case in which $P$ is the identity, as mapping the
identity is not a \emph{definitional} no-op. This subtle point returns with
ferocious intensity when it comes to defining the identity and composition of
contextual functors:

\begin{code}[hide]
  CF-Var : {Γ : C.ctx} {A : C.ty} (v : C.IntVar Γ A) →
    CF-tm (C.makeVar v) ≡ D.makeVar (tr𝑉𝑎𝑟 CF-ty v)
  CF-Var {Γ} v =
    CF-tm (C.makeVar v)
      ≡⟨ deriveMap₁ CF-tm (C.𝒾𝒹 Γ) v ⁻¹ ⟩
    derive (CF-tms (C.𝒾𝒹 Γ)) (tr𝑉𝑎𝑟 CF-ty v)
      ≡⟨ (λ i → derive (CF-id i) (tr𝑉𝑎𝑟 CF-ty v)) ⟩
    D.makeVar (tr𝑉𝑎𝑟 CF-ty v)
      ∎

  transpCF-tm : {Γ : C.ctx} {A B : C.ty} (a : A ≡ B) (t : C.tm Γ A) →
    transport (λ i → D.tm (CF-ctx Γ) (CF-ty (a i))) (CF-tm t)
      ≡ CF-tm (transport (λ i → C.tm Γ (a i)) t)
  transpCF-tm {Γ} a t =
    J (λ B a → transport (λ i → D.tm (CF-ctx Γ) (CF-ty (a i))) (CF-tm t)
      ≡ CF-tm (transport (λ i → C.tm Γ (a i)) t))
      (transportRefl (CF-tm t) ∙ ap CF-tm (transportRefl t ⁻¹)) a

  open Functor

  -- A contextual functor induces a functor between the ambient categories

  ambFun : Functor (ambCat 𝒞) (ambCat 𝒟)
  F-ob ambFun = CF-ctx
  F-hom ambFun = CF-tms
  F-id ambFun = CF-id
  F-seq ambFun τ σ = CF-comp σ τ

open Contextual
open ContextualFunctor
\end{code}

\begin{AgdaMultiCode}
\begin{code}
idCF : (𝒞 : Contextual ℓ₁ ℓ₂) → ContextualFunctor 𝒞 𝒞
CF-ty (idCF 𝒞) A = A
CF-tm (idCF 𝒞) {Γ} {A} t = transport (λ i → tm 𝒞 (map𝐶𝑡𝑥id Γ (~ i)) A) t
\end{code}
\begin{code}[hide]
CF-id (idCF 𝒞) {Γ} =
  map𝑇𝑚𝑠₁ (λ {A} t → transport (λ i → tm 𝒞 (map𝐶𝑡𝑥id Γ (~ i)) A) t) (𝒾𝒹 𝒞 Γ)
    ≡⟨ map𝑇𝑚𝑠₁id (𝒾𝒹 𝒞 Γ) ⟩
  transport (λ i → 𝑇𝑚𝑠 (tm 𝒞) (map𝐶𝑡𝑥id Γ (~ i)) (map𝐶𝑡𝑥id Γ (~ i))) (𝒾𝒹 𝒞 Γ)
    ≡⟨ transp𝒾𝒹 𝒞 (map𝐶𝑡𝑥id Γ ⁻¹) ⟩
  𝒾𝒹 𝒞 (map𝐶𝑡𝑥 (λ A → A) Γ)
    ∎
CF-sub (idCF 𝒞) {Γ} {Δ} {A} t σ =
  transport (λ i → C.tm (map𝐶𝑡𝑥id Γ (~ i)) A) (t C.⟦ σ ⟧)
    ≡⟨ C.transp⟦⟧ (map𝐶𝑡𝑥id Γ ⁻¹) (map𝐶𝑡𝑥id Δ ⁻¹) t σ ⟩
  transport (λ i → C.tm (map𝐶𝑡𝑥id Δ (~ i)) A) t
    C.⟦ transport (λ i → C.tms (map𝐶𝑡𝑥id Γ (~ i)) (map𝐶𝑡𝑥id Δ (~ i))) σ ⟧
    ≡⟨ (λ i → transport (λ i → C.tm (map𝐶𝑡𝑥id Δ (~ i)) A) t C.⟦ map𝑇𝑚𝑠₁id σ (~ i) ⟧) ⟩
  transport (λ i → C.tm (map𝐶𝑡𝑥id Δ (~ i)) A) t
    C.⟦ map𝑇𝑚𝑠₁ (λ {B} → transport (λ i → C.tm (map𝐶𝑡𝑥id Γ (~ i)) B)) σ ⟧
    ∎ where
  module C = Contextual 𝒞
\end{code}
\begin{code}

_∘CF_ : {𝒞 : Contextual ℓ₁ ℓ₂} {𝒟 : Contextual ℓ₃ ℓ₄} {ℰ : Contextual ℓ₅ ℓ₆} →
  ContextualFunctor 𝒟 ℰ → ContextualFunctor 𝒞 𝒟 → ContextualFunctor 𝒞 ℰ
CF-ty (G ∘CF F) = CF-ty G ∘ CF-ty F
CF-tm (_∘CF_ {ℰ = ℰ} G F) {Γ} {A} t  =
  transport (λ i → tm ℰ (map𝐶𝑡𝑥comp (CF-ty G) (CF-ty F) Γ i) (CF-ty G (CF-ty F A)))
    (CF-tm G (CF-tm F t))
\end{code}
\end{AgdaMultiCode}

In the case of the identity, we have have to transport $t$ as to map the
identity function to its domain context. In the case of composition, we have to
translate an iterated mapping to a single mapping of a composition. The sheer
existential dread of this fact has been omitted from the above code - the
requisite types of the $\mathsf{CF\mhyphen id}$ and $\mathsf{CF\mhyphen sub}$
fields involve multiple transports, and this squarely lands one in
\emph{transport hell}.

One can also show (although this is not present in the codebase) the left and
right identity laws as well as associativity of composition. (This is not as
hard as one might initially think because our assumption that $\mathsf{tm}$ is
an h-set saves us from proving tricky coherence laws.) We thus have a category
whose objects are contextual categories and whose morphisms are contextual
functors. One may express some concern in regards to the fact that our
constructions were somewhat messy, and thus might not represent the \emph{true}
notion of this categorical structure. \emph{However, at a very high level, some
univalence principle arguments should show that the space of all possible
compositions, satisfying some niceness properties, is contractible.}

\subsection{Summary}

In this section, we have exhibited the general logic of theories which are
constructed from a notion of types and terms, and equipped a notion of
substitution and identities (or, equivalently, variables), provided that the
manner of this construction is absent of dependencies.

\noindent
\emph{[Here I spend some time discussing the relationship between different
notions of categories for type theory.]}

\begin{code}[hide]
CF-id (_∘CF_ {𝒞 = 𝒞} {𝒟} {ℰ} G F) {Γ} =
  map𝑇𝑚𝑠₁ (CF-tm (G ∘CF F)) (𝒾𝒹 𝒞 Γ)
    ≡⟨ map𝑇𝑚𝑠comp₂ (λ {A} → transport (λ i → tm ℰ (map𝐶𝑡𝑥comp (CF-ty G) (CF-ty F) Γ i) A))
      (CF-tm G ∘ CF-tm F) (𝒾𝒹 𝒞 Γ) ⁻¹ ⟩
  map𝑇𝑚𝑠 (λ {A} → transport (λ i → tm ℰ (map𝐶𝑡𝑥comp (CF-ty G) (CF-ty F) Γ i) A))
    (map𝑇𝑚𝑠₁ (λ x → CF-tm G (CF-tm F x)) (𝒾𝒹 𝒞 Γ))
    ≡⟨ map𝑇𝑚𝑠comp₃ (CF-tm G) (CF-tm F) (𝒾𝒹 𝒞 Γ) ⟩
  transport (λ i → tms ℰ (map𝐶𝑡𝑥comp (CF-ty G) (CF-ty F) Γ i) (map𝐶𝑡𝑥comp (CF-ty G) (CF-ty F) Γ i))
    (CF-tms G (CF-tms F (𝒾𝒹 𝒞 Γ)))
    ≡⟨ (λ i → transport (λ i → tms ℰ (map𝐶𝑡𝑥comp (CF-ty G) (CF-ty F) Γ i)
      (map𝐶𝑡𝑥comp (CF-ty G) (CF-ty F) Γ i)) (CF-tms G (CF-id F i))) ⟩
  transport (λ i → tms ℰ (map𝐶𝑡𝑥comp (CF-ty G) (CF-ty F) Γ i) (map𝐶𝑡𝑥comp (CF-ty G) (CF-ty F) Γ i))
    (CF-tms G (𝒾𝒹 𝒟 (CF-ctx F Γ)))
    ≡⟨ (λ i → transport (λ i → tms ℰ (map𝐶𝑡𝑥comp (CF-ty G) (CF-ty F) Γ i)
      (map𝐶𝑡𝑥comp (CF-ty G) (CF-ty F) Γ i)) (CF-id G i)) ⟩
  transport (λ i → tms ℰ (map𝐶𝑡𝑥comp (CF-ty G) (CF-ty F) Γ i) (map𝐶𝑡𝑥comp (CF-ty G) (CF-ty F) Γ i))
    (𝒾𝒹 ℰ (CF-ctx G (CF-ctx F Γ)))
    ≡⟨ transp𝒾𝒹 ℰ (map𝐶𝑡𝑥comp (CF-ty G) (CF-ty F) Γ) ⟩
  𝒾𝒹 ℰ (map𝐶𝑡𝑥 (CF-ty G ∘ CF-ty F) Γ)
    ∎
CF-sub (_∘CF_ {𝒞 = 𝒞} {𝒟} {ℰ} G F) {Γ} {Δ} {A} t σ =
  transport (λ i → tm ℰ (map𝐶𝑡𝑥comp (CF-ty G) (CF-ty F) Γ i) (CF-ty G (CF-ty F A)))
    (CF-tm G (CF-tm F (t C.⟦ σ ⟧)))
    ≡⟨ (λ i → transport (λ i → tm ℰ (map𝐶𝑡𝑥comp (CF-ty G) (CF-ty F) Γ i) (CF-ty G (CF-ty F A)))
      (CF-tm G (CF-sub F t σ i))) ⟩
  transport (λ i → tm ℰ (map𝐶𝑡𝑥comp (CF-ty G) (CF-ty F) Γ i) (CF-ty G (CF-ty F A)))
    (CF-tm G (CF-tm F t D.⟦ CF-tms F σ ⟧))
    ≡⟨ (λ i → transport (λ i → tm ℰ (map𝐶𝑡𝑥comp (CF-ty G) (CF-ty F) Γ i) (CF-ty G (CF-ty F A)))
      (CF-sub G (CF-tm F t) (CF-tms F σ) i)) ⟩
  transport (λ i → tm ℰ (map𝐶𝑡𝑥comp (CF-ty G) (CF-ty F) Γ i) (CF-ty G (CF-ty F A)))
    (CF-tm G (CF-tm F t) E.⟦ CF-tms G (CF-tms F σ) ⟧)
    ≡⟨ E.transp⟦⟧ (map𝐶𝑡𝑥comp (CF-ty G) (CF-ty F) Γ) (map𝐶𝑡𝑥comp (CF-ty G) (CF-ty F) Δ)
      (CF-tm G (CF-tm F t)) (CF-tms G (CF-tms F σ)) ⟩
  transport (λ i → E.tm (map𝐶𝑡𝑥comp (CF-ty G) (CF-ty F) Δ i) (CF-ty G (CF-ty F A)))
    (CF-tm G (CF-tm F t)) E.⟦ transport (λ i → E.tms (map𝐶𝑡𝑥comp (CF-ty G) (CF-ty F) Γ i)
      (map𝐶𝑡𝑥comp (CF-ty G) (CF-ty F) Δ i)) (CF-tms G (CF-tms F σ)) ⟧
    ≡⟨ (λ i → transport (λ i → E.tm (map𝐶𝑡𝑥comp (CF-ty G) (CF-ty F) Δ i) (CF-ty G (CF-ty F A)))
      (CF-tm G (CF-tm F t)) E.⟦ map𝑇𝑚𝑠comp₃ (CF-tm G) (CF-tm F) σ (~ i) ⟧) ⟩
  transport (λ i → E.tm (map𝐶𝑡𝑥comp (CF-ty G) (CF-ty F) Δ i) (CF-ty G (CF-ty F A)))
    (CF-tm G (CF-tm F t)) E.⟦ map𝑇𝑚𝑠 (λ {B} → transport
      (λ i → E.tm (map𝐶𝑡𝑥comp (CF-ty G) (CF-ty F) Γ i) B)) (map𝑇𝑚𝑠₁ (CF-tm G ∘ CF-tm F) σ) ⟧
    ≡⟨ (λ i → transport (λ i → E.tm (map𝐶𝑡𝑥comp (CF-ty G) (CF-ty F) Δ i) (CF-ty G (CF-ty F A)))
      (CF-tm G (CF-tm F t)) E.⟦ map𝑇𝑚𝑠comp₂ {tm₂ = E.tm} (λ {B} → transport
        (λ i → E.tm (map𝐶𝑡𝑥comp (CF-ty G) (CF-ty F) Γ i) B)) (CF-tm G ∘ CF-tm F) σ i ⟧) ⟩
  transport (λ i → E.tm (map𝐶𝑡𝑥comp (CF-ty G) (CF-ty F) Δ i) (CF-ty G (CF-ty F A)))
    (CF-tm G (CF-tm F t)) E.⟦ map𝑇𝑚𝑠₁ (λ {B} s → transport
      (λ i → E.tm (map𝐶𝑡𝑥comp (CF-ty G) (CF-ty F) Γ i) (CF-ty G (CF-ty F B)))
      (CF-tm G (CF-tm F s))) σ ⟧
    ∎ where
    module C = Contextual 𝒞
    module D = Contextual 𝒟
    module E = Contextual ℰ
\end{code}
