\begin{code}[hide]
{-# OPTIONS --cubical #-}

module NbE where

open import Agda.Builtin.Char public

open import lists

infixr 20 _⇒_
\end{code}

\section{Normalisation by Evaluation}

In this section, we will develop, from first principles, a \emph{Normalisation
by Evaluation} algorithm. This will serve to motivate the notion of
\emph{Kripke} or \emph{presheaf} semantics. Further, the algorithm presented
here will be a minimal computational extraction of the normalisation procedure
evinced by our broader work, so understanding the \emph{why} of how it works
will serve as a motivating question. In what follows, we will borrow some of
the exposition in Abel's thesis.

\subsection{Syntax and Evaluation}

\begin{code}
data Ty : Type where
  Base : Ty
  _⇒_ : Ty → Ty → Ty

Ctx = 𝐶𝑡𝑥 Ty
Var = 𝑉𝑎𝑟 Ty
Ren = 𝑅𝑒𝑛 Ty

data Tm : Ctx → Ty → Type where
  V : {Γ : Ctx} {A : Ty} → Var Γ A → Tm Γ A
  Lam : {Γ : Ctx} {A B : Ty} (t : Tm (Γ ⊹ A) B) → Tm Γ (A ⇒ B)
  App : {Γ : Ctx} {A B : Ty} (t : Tm Γ (A ⇒ B)) (s : Tm Γ A) → Tm Γ B
\end{code}

\clearpage

The above code represents the traditional conception of STLC types and terms.
It does not however, constitute an implementation of STLC, as we have not yet,
for example, defined substitution, nor yet shown that this gives rise to a
CCC category. In fact, this set of terms \emph{is not} the set of terms of any
STLC implementation, because we have not factored into this type the $\beta$
and $\eta$ laws, giving us propositional equalities between syntactically
different, but computationally related, terms. For this, use of something like
a higher inductive type is necessary.

However, for the time being, let us imagine that we have an initial CCC category
whose types are $\mathsf{Ty}$ and whose terms are $\mathsf{Tm}$. We currently
have no interpretation for laws such as $\mathit{app}\beta$, but we have
$\Lambda = \mathsf{Lam}$ and $\mathit{app} = \mathsf{App}$. The base type of
this theory is parametrised by $\top$, where $\mathsf{base} : \top \to
\mathsf{Ty}$ is given by $\mathsf{base}~tt = \mathsf{Base}$.

Given any other CCC category, with a marked off base type $X$, there must, by
definition, be a propositionally unique base and CCC preserving contextual
functor from syntax to this CCC category. This is known as \textbf{evaluation}.
To propositional equality, this functor must have the form:

\begin{align*}
& \lbkt{~A~} & &: \mathsf{ty} \\
& \lbkt{~\mathsf{Base}~} & &= X \\
& \lbkt{~A \Rightarrow B~} & &= \lbkt{~A~} \Rrightarrow \lbkt{~B~} \\
\\
& \lbkt{~t~} & &: \mathsf{tm}~\lbkt{~\Gamma~}~\lbkt{~A~} \\
& \lbkt{~\mathsf{V}~v~} & &= \mathsf{makeVar}~v \\
& \lbkt{~\mathsf{Lam}~t~} & &= \Lambda~\lbkt{~t~} \\
& \lbkt{~\mathsf{App}~t~s~} & &= \mathit{app}~\lbkt{~t~}~\lbkt{~s~}
\end{align*}

\subsection{Sets} \label{section:Sets}

The usual notion of set theory and functions gives rise to a cartesian closed
category. This gives rise to a CCC via the enveloping category construction. In
this case, if we specify a base set $X$, we get an interpretation of syntax into
set theory. Terms of type $A$ in context $\Gamma$ are interpreted as functions
from the folded product $\Downarrow\!\!\mathsf{Ctx}~\lbkt{~\Gamma~}$, known as
an \emph{evaluation context}, to the set $\lbkt{~A~}$. If we have a variable $v
: \mathsf{Var}~\Gamma~A$, and evaluation context $\rho : \Downarrow\!\!
\mathsf{Ctx}~\lbkt{~\Gamma~}$, then we will write $\mathsf{derive}~\rho~v :
\lbkt{~A~}$ for the evident indexing. The evaluation functor becomes:

\begin{align*}
& \lbkt{~A~} & &: \mathsf{Set} \\
& \lbkt{~\mathsf{Base}~} & &= X \\
& \lbkt{~A \Rightarrow B~} & &= \lbkt{~A~} \to\lbkt{~B~} \\
\\
& \lbkt{~t~}~(\rho :~ \Downarrow\!\!\mathsf{Ctx}~\lbkt{~\Gamma~}) & &: \lbkt{~A~} \\
& \lbkt{~\mathsf{V}~v~}~\rho & &= \mathsf{derive}~\rho~v \\
& \lbkt{~\mathsf{Lam}~t~}~\rho & &= a \mapsto \lbkt{~t~}~(\rho,~a) \\
& \lbkt{~\mathsf{App}~t~s~}~\rho & &= \bkt{\lbkt{~t~}~\rho}~\bkt{\lbkt{~s~}~\rho}
\end{align*}

\subsection{Normals and Neutrals}

Terms, although this is not evident in our syntax, are subject to computational
laws. These laws generate an equivalence relation on terms, giving rise to
a notion of two terms being computationally related. Our present goal is to
define a notion of \textbf{normal forms}, which are to be \emph{a representation
of} distinguished members in each class of terms. Our present sole consideration
is to ensure that no class of terms be represented by multiple normal forms.
That each class of terms has a normal form will follow from the normalisation
theorem. The reason that we \emph{currently} care about this property is that it
establishes the decidable equality of terms by way of comparing normal forms.

As mentioned above, we have not fully furnished our syntax with the structure
of a contextual category, so we don't have a definition of substitution.
Further, working with De Bruijn variables will require a notion of weakening
in order to state the $\eta$ law. This is not important for the considerations
of working out what is a reduced form, and so we shall employ informal lambda
calculus notation with named variables in what follows.

The two laws to which our system is subject are the $\beta$-law, which states
that $\bkt{\lambda x. t}~s \equiv t~[x \mapsto s]$, and the $\eta$-law, which
states that, whenever $f$ is of arrow type, that $f \equiv \lambda x. f~x$.
Intuitively, in order to obtain a normal form, one should repeatedly
$\beta$-reduce whenever possible, and then $\eta$-expand whenever doing so will
not result in a $\beta$-redex. We shall not actually perform this procedure, but
we shall define an inductive type that captures its \emph{stopping points}. One
then wonders if any class of terms has multiple stopping points. That this does
not occur is guaranteed by a \emph{confulence theorem}. However, as it turns
out, we will eventually get all the utility of a normalisation theorem without
proving this! Consequently, we presently rely on the unproven assumption of
confluence in order to intuitively guarantee that our notion of normal forms is
useful. Subsequently, we will explain how out notion of normal forms can be
useful without ever establishing that each class of terms has at most one
normal form.

We first address the $\beta$-law. To this end, we want a notion of normal forms
such that no normal form represent a term with a $\beta$-redex, that is,
any subexpression of the form $\bkt{\lambda x. t}~s$. The main issue that we
run into is that the class of $\beta$-redex-free terms is not closed under
application. For example, $a : \mathsf{Base} \vdash (\lambda x. x)~a :
\mathsf{Base}$ is a reducible term despite the fact that $\lambda x. x$ and the
variable $a$ are individually redex free.

Now, whenever we have an application, $t~s$, $t$ must be or arrow type and thus
represents a function. Why then is it not always the case that an application
can always be evaluated? The answer is that the function could be an
\emph{unknown placeholder} for a function, that is, a variable. Alternatively,
if $x$ is a variable of type $A \to B \to C$, and $t$ is any redex free
expression of type $A$, then $x~t$ is also redex free and is not of the form of
a lambda. By induction, one establishes that any redex free expression which is
not of the form of a lambda has the form of a variable, possibly with any number
of redex-free arguments applied to it. This gives rise to the notion of a
\textbf{neutral form}. These are to thought of as \emph{generalised variables},
in the sense that they are maximally blocked of from any further evaluation due
to a lack of information.

At this point, we could say that all neutral forms are normal. We would then
have variable formation and application be neutral form constructors, and
neutral reflection and lambda be normal form constructors. Through mutual
induction, we would get a definition of $\beta$-reduced normal forms. This,
however, does not account for $\eta$-expansion.

Where, in a $\beta$-reduced term would we perform $\eta$-expansion? Well, if
we eta expand a lambda term, we obtain $\lambda a. t = \lambda x. (\lambda a .
t)~x = \lambda x. \bkt{t~[a \mapsto x]} = \lambda a. t$, that is,
$\eta$-expansion leads to a $\beta$-reducible expression that converts into
an $\alpha$-equivalent form of our starting point, so this creates a dead loop.
Next, if we have a neutral expression of the form $x~t_1~\ldots~t_n$, then
converting the variable to a lambda expression again creates a $\beta$-redex
that gives rise to a dead loop, this also applies to $\eta$-expanding any
entire neutral expression of the form $x~t_1~\ldots~t_n$ whenever it is
nested inside some further application $x~t_1~\ldots~t_n~t_{n+1}$. These two
points, however, noes not address $\eta$-expanding $t$ inside of the expression
$\lambda a. t$, or $\eta$-expanding a topmost neutral expression
$x~t_1~\ldots~t_n$.

We can now see how $\eta$-expansion is going to work. Suppose that we start with
a $\beta$-reduced term $t$. First off, we address what it means for a term to be
$\eta$ expanded at the topmost level. Suppose that $t$ has type $A_n \to \ldots
\to A_1 \to \mathsf{Base}$. If $t$ is not a lambda, then we $\eta$-expand the
expression. If not, or following the first step, $t$ is of the form $\lambda x.
s$, for $s : A_{n-1} \to \ldots \to A_1 \to B$. We then repeatedly do a mixture
of $\eta$ expanding and recursing inside of lambdas until we end up at an
expression of type $\mathsf{Base}$. This is not an arrow type, and so we can't
$\eta$-expand any further. At no point in the process will we have created new
$\beta$-redexes. Now, an expression of base type cannot be a lambda, as lambdas
have arrow type. Further, this expression is to be $\beta$-reduced, and is
consequently a neutral term.

A consequence of the above exposition is that any normal form of type $A_n \to
\ldots \to A_1 \to \mathsf{Base}$ is of the form $\lambda x_n. \cdots \lambda
x_1. t$ where $t$ is a neutral form of $\mathsf{Base}$ type. Now, $t$ is of the
form $x~t_1~\ldots~t_m$. In order to ensure that all possible non-cyclic $\eta$
expansions have been met, we recursively apply this entire process to each
expression $t_i$, which eventually yields a normal form. (Note that if $t_i$
is one one the variables introduced by an $\eta$ expansion, then it will not
be normal unless it is of base type.)

All together, we see that we are to amend our notion of neutral terms such that
the $t_i$ applied to a variable are to be normal, and not merely $\beta$-redex
free. We also see that neutral terms of base type are normal. That we have taken
every opportunity to $\eta$-expand is witnessed by the fact that its fruitless
to $\eta$-expand lambdas or neutral forms on the left side of an application. A
static description, then of what it means for a term to be $\beta$-reduced and
$\eta$-long is achieved by the following mutually inductive definitions:

\begin{code}
data Nf : Ctx → Ty → Type
data Ne : Ctx → Ty → Type

data Ne where
  VN : {Γ : Ctx} {A : Ty} → Var Γ A → Ne Γ A
  APP : {Γ : Ctx} {A B : Ty} → Ne Γ (A ⇒ B) → Nf Γ A → Ne Γ B

data Nf where
  NEU : {Γ : Ctx} → Ne Γ Base → Nf Γ Base
  LAM : {Γ : Ctx} {A B : Ty} → Nf (Γ ⊹ A) B → Nf Γ (A ⇒ B)
\end{code}

\subsection{A First Attempt At Normalisation}

We will presently derive a normalisation algorithm under the incorrect
assumption that we can entirely ignore the presence of the context $\Gamma$ in
the type of normal and neutral forms. What this means is that we will suppose
that we have sets $\mathsf{Nf}~A$ and $\mathsf{Ne}~A$ for $A : \mathsf{Ty}$,
and that all of the syntactic formers for these forms apply in a way oblivious
to contexts. So, for example, if $M : \mathsf{Ne}~(A \Rightarrow B)$ and $N :
\mathsf{Nf}~A$, then $\mathsf{APP}~M~N : \mathsf{Ne}~B$.

Following the description in Section \ref{section:Sets}, we shall eliminate
syntax into sets, setting $\lbkt{~\mathsf{Base}~} = \mathsf{Nf}~\mathsf{Base}$.
Members of the set $\lbkt{~A~}$ are known as \textbf{semantic elements} of type
$A$. As for the interpretation of terms, $\lbkt{~t~} : \Downarrow\!\!
\mathsf{Ctx}~ \lbkt{~\Gamma~} \to \lbkt{~A~}$. Supposing that we could cook up
the nested pairs of semantic elements to the left, applying $\lbkt{~t~}$ would
give us a semantic element of type $A$. We would then need to turn this into
a normal form. The key to normalisation, then is a method for passing from
normal/neutral forms to semantic elements and back.

The terminology for these constructions comes from LISP. There, a piece of code
that is part of a program is generally something that gets executed and produces
a value, and is thus a semantic expression. LISP also has an internal
S-expression representation of its syntax. Code can be \emph{quoted} into
syntax, at which point computations (possibly dependant on user input) can be
performed on the result. Finally, one can \emph{unquote} syntax into a program,
which then gets evaluated. This process is known as \emph{meta-circular
evaluation}.

To this end, we define functions  $\mathsf{q}_A : \lbkt{~A~} \to \mathsf{Nf}~A$
and $\mathsf{u}_A : \mathsf{Ne}~A \to \lbkt{~A~}$, called \textbf{quote} and
\textbf{unquote}. These are defined mutually, by induction on the type $A$:
\begin{align*}
\mathsf{q}_\mathsf{Base}~(N : \mathsf{Nf}~\mathsf{Base}) &= N \\
\mathsf{q}_{A \Rightarrow B} ~ (\mathcal{f} : \lbkt{~A~} \to \lbkt{~B~}) &=
\mathsf{LAM}~\bkt{\mathsf{q}_B~\bkt{\mathcal{f}~\bkt{\mathsf{u}_A~(\mathsf{VN}
~\mathit{zv})}}} \\
\\
\mathsf{u}_\mathsf{Base}~(M : \mathsf{Ne}~\mathsf{Base}) &= \mathsf{NEU}~M \\
\mathsf{u}_{A \Rightarrow B}~(M : \mathsf{Ne}~(A \Rightarrow B)) &=
(\mathcal{s} : \lbkt{~A~}) \mapsto \mathsf{u}_B~\bkt{\mathsf{APP}~M~\bkt{
\mathsf{q}_A~\mathcal{s}}}
\end{align*}

At the base type, the requirements that we ought to satisfy aren't to difficult.
For quote, both the domain and codomain are $\mathsf{Nf}~\mathsf{Base}$. For
unquote, we use the $\mathsf{NEU}$ constructor to turn a $\mathsf{Ne}
~\mathsf{Base}$ into a $\mathsf{Nf}~\mathsf{Base}$.

At arrow types, the situation is more subtle. For quote, we begin with a
function $\mathcal{f} : \lbkt{~A~} \to \lbkt{~B~}$. If this is to be used, this
is to be feed it a member of $\lbkt{~A~}$. In general, we can use unquote to
produce a semantic element, but for this we need a neutral form of type $A$.
Such a term can always be produced if we assume that $A$ is in our context, and,
since we're currently ignoring contexts, \emph{why not?} We thus simply use a
variable. With this, we get a semantic element in $\lbkt{~B~}$, which we unquote
to get a normal form of type $B$. We then abstract to get a normal form of type
$A \Rightarrow B$.

Unquote is less clever; we begin with a neutral term of type $A \Rightarrow B$,
and we are to produce a function of $\lbkt{~A~} \to \lbkt{~B~}$. Then, given
$\mathcal{s}$ in $\lbkt{~A~}$, we can quote it to get a normal form of type $A$.
We then use $\mathsf{APP}$ to get a neutral term of type $B$, and we can unquote
thus to get a semantic element in $\lbkt{~B~}$. Observe that, in order to have
formed the application, we needed $M$ to be neutral, explaining why the domain
of unquote is restricted to this class.

We should note that the defenitions of these functions have the present property
that each term former appears exactly once, and that each of quote an unquote
is applied exactly once on the the domain and codomain of an arrow type.

We extend unquote to be defined on substitutions, such that $\mathsf{u}_\Gamma
~\sigma : \Downarrow\!\!\mathsf{Ctx}~ \lbkt{~\Delta~}$. We then take the
variables $\mathsf{id}\mathit{Ren}~\Gamma$ and apply $\mathsf{VN}$ to get
neutral forms. We will use $\mathsf{idNe}~\Gamma$ for this result. Finally, we
have:
\[\mathsf{q}_A~\bkt{\lbkt{~t~}~\bkt{\mathsf{u}_\Gamma~(\mathsf{idNe}~\Gamma)}}
: \mathsf{Nf}~A\]

\subsection{Accounting For Contexts}

When we take contexts into account, $\mathsf{Nf}~A$ splits apart into sets
$\mathsf{Nf}~\Gamma~A$ for each context $\Gamma$, and so our definition of
$\lbkt{~\mathsf{Base}~} = \mathsf{Nf}~A$ no longer makes sense. We won't worry
about the categorical meaning of this for the moment, but what we want to do,
then, is replace the notion of sets with families of context indexed sets.
We thus have that $\lbkt{~A~} : \mathsf{Ctx} \to \mathsf{Set}$, and we shall
write $\lbkt{~A~}_\Gamma$ for the set at $\Gamma$. At the base type, we have
$\lbkt{~\mathsf{Base}~}_\Gamma = \mathsf{Nf}~\Gamma~\mathsf{Base}$.
\emph{Naively}, we set $\lbkt{~A\Rightarrow B~}_\Gamma = \lbkt{~A~}_\Gamma \to
\lbkt{~B~}_\Gamma$, although this will turn out to be the wrong definition to
use.

Quote and unquote are then refined as to also be indexed by a context
$\mathsf{q}_{A,~\Gamma} : \lbkt{~A~}_\Gamma \to \mathsf{Nf}~\Gamma~A$ and
$\mathsf{u}_{A,~\Gamma} : \mathsf{Ne}~\Gamma~A \to \lbkt{~A~}_\Gamma$. At this
point, we can see the main slight of hand that occurred in the previous section.
In the definition of $\mathsf{q}_{A \Rightarrow B,~\Gamma}$, we will now have
that the argument $\mathcal{f}$ has type $\lbkt{~A~}_\Gamma \to
\lbkt{~B~}_\Gamma$. The variable which we used, on the other hand, naturally
lived in $\mathsf{Ne}~(\Gamma~\hermitmatrix~A)~A$, and would thus unquote
to a semantic element in $\lbkt{~A~}_{(\Gamma~\hermitmatrix~A)}$. In order to
make our definition go through, then, we need to be able to evaluate
$\mathcal{f}$ on this element in a more general context.

Towards this end, let's suppose that $\mathcal{f}$ is of the form $\mathsf{u}_{A
\Rightarrow B,~\Gamma}~M$ where $M : \mathsf{Ne}~\Gamma~ (A \Rightarrow B)$. We
have that $\mathsf{u}_{A \Rightarrow B,~\Gamma}~M~\mathcal{s} = \mathsf{u}
_{B,~\Gamma}~\bkt{\mathsf{APP}~M~\bkt{\mathsf{q}_{A,~\Gamma}~\mathcal{s}}}$.
Suppose that we instead take $\mathcal{s} : \lbkt{~A~}_{(\Gamma~\hermitmatrix
~A)}$. We would then unquote $\mathcal{s}$ in context $\Gamma~\hermitmatrix~A$
and attempt to apply it to $M$. \emph{Can $M$ be made into a neutral form in
context $\Gamma~\hermitmatrix~A$?} The answer is yes!

In general, the context $\Gamma$ is pertinent to a term (whether syntactic,
normal, or neutral) in so far as it gives meaning to variables. If we have a
renaming $\mathsf{Ren}~\Delta~\Gamma$, then that specifies how each variable
in $\Gamma$ can be replaced with a variable in $\Delta$. We should thus able to
pull back syntax across a renaming. We do so in code:
\begin{code}
_[_]Ne : {Δ Γ : Ctx} {A : Ty} → Ne Γ A → Ren Δ Γ → Ne Δ A
_[_]Nf : {Δ Γ : Ctx} {A : Ty} → Nf Γ A → Ren Δ Γ → Nf Δ A

VN v [ σ ]Ne = VN (derive σ v)
APP M N [ σ ]Ne = APP (M [ σ ]Ne) (N [ σ ]Nf)

NEU M [ σ ]Nf = NEU (M [ σ ]Ne)
LAM {A = A} N [ σ ]Nf = LAM (N [ W₂𝑅𝑒𝑛 A σ ]Nf)
\end{code}

In general, if we feed $\mathsf{u}_{A \Rightarrow B,~\Gamma}~M$ an element
$\mathcal{s} : \lbkt{~A~}_\Delta$, along with a renaming $\sigma : \mathsf{Ren}
~\Delta~\Gamma$, then we are able to define:
\[\mathsf{u}_{A \Rightarrow B,~\Gamma}~M~\sigma~\mathcal{s} = \mathsf{u}
_{B,~\Delta}~\bkt{\mathsf{APP}~(M~[~\sigma~]\mathsf{Ne})~\bkt{\mathsf{q}
_{A,~\Delta}~\mathcal{s}}} : \lbkt{~B~}_\Delta\]

We see, then, that we need to strengthen the power of semantic elements at arrow
types. Instead of having $\lbkt{~A\Rightarrow B~}_\Gamma = \lbkt{~A~}_\Gamma \to
\lbkt{~B~}_\Gamma$, we set $\lbkt{~A\Rightarrow B~}_\Gamma = \mathsf{Ren}~\Delta
~\Gamma \to \lbkt{~A~}_\Delta \to \lbkt{~B~}_\Delta$.
\begin{code}
El : Ctx → Ty → Type
El Γ Base = Nf Γ Base
El Γ (A ⇒ B) = {Δ : Ctx} → Ren Δ Γ → El Δ A → El Δ B
\end{code}
With this, we are able to complete the definitions of quote and unquote:
\begin{code}
q : {Γ : Ctx} {A : Ty} → El Γ A → Nf Γ A
u : {Γ : Ctx} {A : Ty} → Ne Γ A → El Γ A

q {A = Base} N = N
q {Γ} {A ⇒ B} 𝒻 = LAM (q (𝒻 (W₁𝑅𝑒𝑛 A (id𝑅𝑒𝑛 Γ)) (u (VN 𝑧𝑣))))

u {A = Base} M = NEU M
u {A = A ⇒ B} M σ 𝓈 = u (APP (M [ σ ]Ne) (q 𝓈))
\end{code}
In particular, the definition of quote is now able to feed $\mathcal{f}$ an
element in a stronger context by way of providing a renaming
$(\Gamma~\hermitmatrix~A)$ to $\Gamma$. Quoting results in something of type
$\mathsf{Nf}~(\Gamma~\hermitmatrix~A)~B$, which is perfect for applying
$\mathsf{LAM}$.

It remains to define evaluation. For $t : \mathsf{Tm}~\Gamma~A$, we want to
have that $\lbkt{~t~} : \Downarrow\!\!\mathsf{Ctx}~\lbkt{~\Gamma~} \to
\lbkt{~A~}$. $\lbkt{~t~} : \Downarrow\!\!\mathsf{Ctx}~\lbkt{~\Gamma~}$ is just
supposed to be a folded up version of each semantic set family $\lbkt{~A_i~}$
for $A_i$ a type in $\Gamma$. This folded up object is still supposed to
be a semantic set family, so if we ask what $\bkt{\Downarrow\!\!\mathsf{Ctx}
~\lbkt{~\Gamma~}}_\Delta$ is to be, then it is most reasonable to say that
it is a folded up version of the sets $\lbkt{~A_i~}_\Delta$. As presented
previously, this would be an iterated cartesian product of sets, however, as
far as out minimal code implementation is concerned, we can replace this
with a flattened out indexed list; we set $\bkt{\Downarrow\!\!\mathsf{Ctx}
~\lbkt{~\Gamma~}}_\Delta = \mathsf{Els}~\Delta~\Gamma$, where:
\begin{code}
Els : Ctx → Ctx → Type
Els Δ Γ = 𝑇𝑚𝑠 El Δ Γ
\end{code}

What then, is a morphism $\mathcal{A} \to \mathcal{B}$ of semantic set families?
The most reasonable option is to say that it is a family of functions
$\mathcal{A}_\Delta \to \mathcal{B}_\Delta$ for each context $\Delta$. We
would then have morphisms $\lbkt{~t~}_\Delta : \mathsf{Els}~\Delta~\Gamma \to
\mathsf{El}~\Delta~A$ for each context $\Delta$.

As we try to define this, we come upon the case $\lbkt{~\mathsf{Lam}~t~}$, for
$t : \mathsf{Tm}~\bkt{\Gamma ~\hermitmatrix~A}~B$. For $\mathcal{s}s :
\mathsf{Tms}~\Delta~\Gamma$, we need $\lbkt{~\mathsf{Lam}~t~}~\mathcal{s}s :
\mathsf{El}~\Delta~(A \Rightarrow B)$. This type is by definition $\mathsf{Ren}
~\Sigma~\Delta \to \mathsf{El}~\Sigma~A \to \mathsf{El}~\Sigma~B$. We thus
assume that $\sigma : \mathsf{Ren}~\Sigma~\Delta$ and $\mathcal{s} : \mathsf{El}
~\Sigma~A$, and want to produce an element of type $\mathsf{El}~\Sigma~B$.
Fortunately, our inductive hypothesis lets us do this as $\lbkt{~t~}_\Sigma :
\mathsf{Els}~\Sigma ~(\Gamma~\hermitmatrix~A) \to \mathsf{El}~\Sigma~B$. So we
just need to assemble a list of elements $\mathsf{Els}~\Sigma~(\Gamma~
\hermitmatrix~A)$ given $\mathcal{s}s : \mathsf{Tms}~\Delta~\Gamma$ and
$\mathcal{s} : \mathsf{El} ~\Sigma~A$. In order to do this, we need a method or
renaming semantic elements:
\begin{code}
_[_]El : {Δ Γ : Ctx} {A : Ty} → El Γ A → Ren Δ Γ → El Δ A
_[_]El {A = Base} N σ = N [ σ ]Nf
_[_]El {A = A ⇒ B} 𝒻 σ τ 𝓈 = 𝒻 (σ ∘𝑅𝑒𝑛 τ) 𝓈
\end{code}
With this out of the way, we are able to complete the definition of evaluation:
\begin{code}
eval : {Γ Δ : Ctx} {A : Ty} → Tm Δ A → Els Γ Δ → El Γ A
eval (V v) 𝓈s = derive 𝓈s v
eval (Lam t) 𝓈s σ 𝓈 = eval t (map𝑇𝑚𝑠 _[ σ ]El 𝓈s ⊕ 𝓈)
eval {Γ} (App t s) 𝓈s = eval t 𝓈s (id𝑅𝑒𝑛 Γ) (eval s 𝓈s)
\end{code}
Note that this definition bears striking similarity to the evaluation procedure
for sets described in Section \ref{section:Sets}, albeit modified to take
contexts into account.

We thus have $\lbkt{~t~}_\Gamma = \lbkt{~\Gamma~}_\Gamma \to \lbkt{~\Gamma~}_A$.
As with the previous section, in context $\Gamma$, we have all of the variables
that constitute $\mathsf{id}\mathit{Ren}~\Gamma$. We can turn these into
neutral terms and then unquote as to get a list of semantic elements. We feed
this into $\lbkt{~t~}_\Gamma$ and then quote to obtain a normal form:
\begin{code}
norm : {Γ : Ctx} {A : Ty} → Tm Γ A → Nf Γ A
norm {Γ} t = q (eval t (map𝑇𝑚𝑠 (u ∘ VN) (id𝑅𝑒𝑛 Γ)))
\end{code}
We have thus constructed a normalisation algorithm in 26 lines of code! (44 if
you include the definitions of syntax and normal forms, although this count
excludes the definitions of the helper functions defined in the section on
contextual categories.)

\subsection{A Demo}

First, we give an embedding of normal forms into syntax:
\begin{code}
ιNe : {Γ : Ctx} {A : Ty} → Ne Γ A → Tm Γ A
ιNf : {Γ : Ctx} {A : Ty} → Nf Γ A → Tm Γ A

ιNe (VN v) = V v
ιNe (APP M N) = App (ιNe M) (ιNf N)

ιNf (NEU M) = ιNe M
ιNf (LAM N) = Lam (ιNf N)
\end{code}
Now, we define some elementary Church arithmetic:
\begin{code}
ChurchType : Ty → Ty
ChurchType A = (A ⇒ A) ⇒ A ⇒ A

ChurchTwo : {Γ : Ctx} {A : Ty} → Tm Γ (ChurchType A)
ChurchTwo = Lam (Lam (App (V (𝑠𝑣 𝑧𝑣)) (App (V (𝑠𝑣 𝑧𝑣)) (V 𝑧𝑣))))

PlusType : Ty → Ty
PlusType A = ChurchType A ⇒ ChurchType A ⇒ ChurchType A

Plus : {Γ : Ctx} {A : Ty} → Tm Γ (PlusType A)
Plus = Lam (Lam (Lam (Lam (App (App (V (𝑠𝑣 (𝑠𝑣 (𝑠𝑣 𝑧𝑣)))) (V (𝑠𝑣 𝑧𝑣)))
                               (App (App (V (𝑠𝑣 (𝑠𝑣 𝑧𝑣))) (V (𝑠𝑣 𝑧𝑣))) (V 𝑧𝑣))))))
\end{code}
At which point we can define what, in the development of this project,
served as my primary testcase:
\begin{code}
2+2 : Tm ∅ (ChurchType Base)
2+2 = ιNf (norm (App (App Plus ChurchTwo) ChurchTwo))
\end{code}
At this point one types \texttt{C-c C-n 2+2 RET} and sees that this normalises
to the Church numeral for four. Presented statically (and checked in the process
of compiling this paper):
\begin{code}
2+2=4 : 2+2 ≡ Lam (Lam (App (V (𝑠𝑣 𝑧𝑣)) (App (V (𝑠𝑣 𝑧𝑣))
                  (App (V (𝑠𝑣 𝑧𝑣)) (App (V (𝑠𝑣 𝑧𝑣)) (V 𝑧𝑣))))))
2+2=4 = refl
\end{code}

\subsection{Metatheoretic Utility}

By sole virtue of the fact that we have, in type theory, some term
$\mathsf{norm} : \mathsf{Tm}~\Gamma~A \to \mathsf{Nf}~\Gamma~A$, entails that we
have a terminating algorithm for producing normal forms. One then asks whether
on not such an algorithm produces the \emph{correct} normal forms, in the sense
that $\iota\mathsf{Nf}~(\mathsf{norm}~t) \equiv t$. When phrased in the context
of the above Agda code, this is probably false since non-vacuous normalisation
produces an unequal term, as our inductive type for syntax does not factor in
reduction rules.

To get around this, we can take one of two approaches: The first approach is to
furnish, as a layer above syntax, a type theoretic relation for what it means
for two terms to be computationally equal, and subsequently to amend the
statement of correctness to to refer to relatedness instead of equality. (A set
equipped with an equivalence relation is known as a \emph{setoid}.) This setoid
approach will be used in our \emph{Stepping by Evlauation} algorithm presented
at the end of the paper.

To give you an idea of how one would generate such a relation from the
computation rules, think of the various properties that equality satisfies. We
start off with some syntactic representation of our base rules, but these rules
alone do not witness all related pairs. For one, symmetry and transitivity force
us to upgrade our type of relations to store lists of rules and their formal
inverses. However, this is not all; the ability to \emph{\textbf{ap}}ply a
function to an equality reflects the fact that computation rules can apply in
some subexpression of a term. We are thus forced to replace the occurrence of
rules above by pairs of a rule and an \emph{environment} in which the rule is
being applied. This description yields the type of \textbf{computational
traces}. Such traces can be formatted in a very presentable way to express how
two terms are related by a sequence of applications of computation rules (or
their inverses) to indicated subexpressions.

The second approach is to define the syntax of type theory as a \emph{Higher
Inductive Type}, in which the reduction rules are automatically propositional
equalities. The laborious construction of relatedness above is replaced with
the built-in type of propositional equality, and we can see that all of the
related pairs of terms constructed above could instead be witnessed by
applications of the higher groupoid laws of equality. The disadvantage to this
is that something like transitivity of equality would, in Cubical Agda, be
witnessed by an $\mathsf{hcomp}$, thus, in the least, one would expect computed
proofs that $\iota\mathsf{Nf}~(\mathsf{norm}~t) \equiv t$ to be somewhat
unreadable nested $\mathsf{hcomp}$ expressions \emph{(as it turns out, this is
a sore understatement of the unreadability that we experimentally observe)}.

In order to amend the above code to make terms be a HIT, we first need a notion
of weakening and substitution in order to even state the $\beta$ and $\eta$
laws. However, weakening and substitution are notions defined on syntax, so we
need a notion of terms. This cyclic situation is solved by using
\emph{induction-recursion} to simultaneously define syntax with notions of
weakening and substitution. In the course of these constructions, any inductive
definition on terms must handle the higher constructors; this precisely means
that we must show that our notion of weakening and substitution respect
computational equality. (We will eventually take the alternative approach of
modifying syntax to include explicit substitutions, thus moving the
recursive-induction to plain old induction.)

Following this, we aren't done yet. Now that terms are a HIT, induction over
them becomes more elaborate. The one place where, in the above normalisation
algorithm, we used induction on terms was in the definition of $\mathsf{eval}$.
Accounting for the higher cases does two things: on the surface level, it
tells us how to apply $\mathsf{eval}$ to the higher constructors, but, more
meaningfully, it establishes the coherence of $\mathsf{eval}$ with respect
to equalities, thus entailing that $t \equiv s$ implies $\mathsf{norm}~t \equiv
\mathsf{norm}~s$.

The analogous property in the setoid approach is important. If we refer to the
relation of computational equality as $\mathsf{Steps}$, then it states that
$\mathsf{Steps}~t~s$ implies $\mathsf{norm}~t \equiv \mathsf{norm}~s$. The
contrapositive of this states that terms which do not have the same normal forms
are not computationally equal. On the other hand, if $\mathsf{norm}~t \equiv
\mathsf{norm}~s$, then $\iota\mathsf{Nf}~\bkt{\mathsf{norm}~t} \equiv
\iota\mathsf{Nf}~\bkt{\mathsf{norm}~s}$, and assuming that we've also proven
$\mathsf{Steps}~\bkt{\iota\mathsf{Nf}~(\mathsf{norm}~t)}~t$, we are able to
conclude that $\mathsf{Steps}~t~s$. This same analysis applies to the HIT
situation when we replace $\mathsf{Steps}$ with propositional equality.

Now, the key way in which out type of normal forms is metatheoretically useful
is that, as a discrete inductive type, it has decidable equality. Combining
this with the above, we are able to conclude, in the HIT approach, decidable
equality for terms, and, in the setoid approach, decidability of
$\mathsf{Steps}$.

In particular, the key property that makes normal forms useful \emph{is not} the
fact that every class of terms is represented by at most one normal form! In
fact, one can actually avoid normal forms entirely by setting the type of
semantic elements at $\mathsf{Base}$ to be the type of terms (except that in
the HIT approach, we would remove the presence of relations). The first
modification that one then makes is replacing $\mathsf{\_[\_]Nf}$ with a renaming
operation on terms. The definition of $\mathsf{eval}$ goes through unchanged.
Finally, as it turns out, our definitions of quote and unquote work perfectly
fine when normals and neutrals are replaced with syntax. This yields a function
$\mathsf{norm} : \mathsf{Tm}~\Gamma~A \to \mathsf{Tm}~\Gamma~A$, and running
it on $\mathsf{2+2}$ results in a reduced expression! Morally, what's going on
is that we've englarged the semantic domain (unquotiented syntax instead of
normal forms), but left the algorithm unchanged, so the image of the
$\mathsf{norm}$ function still lands in reduced expressions. This observation,
though, is irrelevant for the decidability of equality, as all that matters is
that we have successfully factored the identity through a discrete type.
